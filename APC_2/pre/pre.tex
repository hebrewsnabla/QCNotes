%%%%%%%%%%%%%%%%%%%%%%%%%%%%%%%%%%%%%%%%%
% Beamer Presentation
% LaTeX Template
% Version 1.0 (10/11/12)
%
% This template has been downloaded from:
% http://www.LaTeXTemplates.com
%
% License:
% CC BY-NC-SA 3.0 (http://creativecommons.org/licenses/by-nc-sa/3.0/)
%
%%%%%%%%%%%%%%%%%%%%%%%%%%%%%%%%%%%%%%%%%

%----------------------------------------------------------------------------------------
%	PACKAGES AND THEMES
%----------------------------------------------------------------------------------------

\documentclass[10pt,aspectratio=43,mathserif]{beamer}
\usepackage{nju}			 %导入 nju 模板宏包
%\usepackage[utf8]{inputenc}
%\usepackage[UTF8]{ctex}
%\usepackage{xeCJK}

\mode<presentation> {

% The Beamer class comes with a number of default slide themes
% which change the colors and layouts of slides. Below this is a list
% of all the themes, uncomment each in turn to see what they look like.

%\usetheme{default}
%\usetheme{AnnArbor}
%\usetheme{Antibes}
%\usetheme{Bergen}
%\usetheme{Berkeley}
%\usetheme{Berlin}
%\usetheme{Boadilla}
%\usetheme{CambridgeUS}
%\usetheme{Copenhagen}
%\usetheme{Darmstadt}
%\usetheme{Dresden}
%\usetheme{Frankfurt}
%\usetheme{Goettingen}
%\usetheme{Hannover}
%\usetheme{Ilmenau}
%\usetheme{JuanLesPins}
%\usetheme{Luebeck}
%\usetheme{Madrid}
%\usetheme{Malmoe}
%\usetheme{Marburg}
%\usetheme{Montpellier}
%\usetheme{PaloAlto}
%\usetheme{Pittsburgh}
%\usetheme{Rochester}
%\usetheme{Singapore}
%\usetheme{Szeged}
%\usetheme{Warsaw}

% As well as themes, the Beamer class has a number of color themes
% for any slide theme. Uncomment each of these in turn to see how it
% changes the colors of your current slide theme.

%\usecolortheme{albatross}
%\usecolortheme{beaver}
%\usecolortheme{beetle}
%\usecolortheme{crane}
%\usecolortheme{dolphin}
%\usecolortheme{dove}
%\usecolortheme{fly}
%\usecolortheme{lily}
%\usecolortheme{orchid}
%\usecolortheme{rose}
%\usecolortheme{seagull}
%\usecolortheme{seahorse}
%\usecolortheme{whale}
%\usecolortheme{wolverine}

%\setbeamertemplate{footline} % To remove the footer line in all slides uncomment this line
%\setbeamertemplate{footline}[page number] % To replace the footer line in all slides with a simple slide count uncomment this line

%\setbeamertemplate{navigation symbols}{} % To remove the navigation symbols from the bottom of all slides uncomment this line
}

\usepackage{graphicx} % Allows including images
\usepackage{booktabs} % Allows the use of \toprule, \midrule and \bottomrule in tables
\usepackage[version=3]{mhchem}
\usepackage{physics}
\setlength{\parindent}{0pt}

%\usepackage{symbols}
%\usepackage{eucal}
\usepackage{mathrsfs}
\usepackage{amsmath,amsfonts,amssymb,bm}   %导入数学公式所需宏包
\usepackage{float}
\usepackage{multicol}
%\usepackage{abstract}
\usepackage{empheq}
\usepackage{extarrows}
%\usepackage{cite}
\usepackage{listings}
\usepackage{multirow}
\usepackage{xcolor}
\usepackage{fontspec}  
\usepackage{braket}
%\usepackage{mathspec}
%\usepackage{unicode-math}
%\usepackage[table,xcdraw]{xcolor}
\usepackage{color}
\usepackage{siunitx}
\usepackage[isbn=false,doi=false,uniquename=init]{biblatex}
\addbibresource{pre.bib}

%\setmonofont{}
\newfontfamily\Consolas{Consolas}
%\setsansfont{Arial} %设置英文字体 Monaco, Consolas,  Fantasque Sans Mono


%\setmainfont{Times New Roman}
%\setsansfont{Arial}
%\setmonofont{Courier New}
\usepackage[indentfirst]{xeCJK}
%\setCJKmainfont[BoldFont={SimHei},ItalicFont={KaiTi}]{SimSun}
%\setCJKsansfont{KaiTi}

%\usepackage{palatino}
\usepackage{tikz}
\usetikzlibrary{shapes.geometric, arrows}

\definecolor{dkgreen}{rgb}{0,0.6,0}
\definecolor{gray}{rgb}{0.5,0.5,0.5}
\definecolor{mauve}{rgb}{0.58,0,0.82}
%\renewcommand\ttdefault{phv}

\lstset{ %
	language=Python,                % the language of the code
	basicstyle=\Consolas\footnotesize,           % the size of the fonts that are used for the code
	numbers=left,                   % where to put the line-numbers
	numberstyle=\tiny\color{gray},  % the style that is used for the line-numbers
	stepnumber=1,                   % the step between two line-numbers. If it's 1, each line 
	% will be numbered
	numbersep=5pt,                  % how far the line-numbers are from the code
	backgroundcolor=\color{white},      % choose the background color. You must add \usepackage{color}
	showspaces=false,               % show spaces adding particular underscores
	showstringspaces=false,         % underline spaces within strings
	showtabs=false,                 % show tabs within strings adding particular underscores
	frame=single,                   % adds a frame around the code
	rulecolor=\color{black},        % if not set, the frame-color may be changed on line-breaks within not-black text (e.g. commens (green here))
	tabsize=4,                      % sets default tabsize to 2 spaces
	captionpos=b,                   % sets the caption-position to bottom
	breaklines=true,                % sets automatic line breaking
	breakatwhitespace=false,        % sets if automatic breaks should only happen at whitespace
	title=\lstname,                   % show the filename of files included with \lstinputlisting;
	% also try caption instead of title
	keywordstyle=\color{blue},          % keyword style
	commentstyle=\color{dkgreen},       % comment style
	stringstyle=\color{mauve},         % string literal style
	escapeinside={\%*}{*)},            % if you want to add LaTeX within your code
	morekeywords={*,...}               % if you want to add more keywords to the set
}

%\DeclareMathOperator{\p}{\prime}
\DeclareMathOperator{\ti}{\times}
\DeclareMathOperator{\s}{\sum_{n=1}^{\infty}}
\DeclareMathOperator{\intinf}{\int_0^\infty}
\DeclareMathOperator{\intdinf}{\int_{-\infty}^\infty}
\DeclareMathOperator{\suminf}{\sum_{n=0}^\infty}
\DeclareMathOperator{\e}{\mathrm{e}}
%\renewcommand{\I}{\mathrm{i}}
\DeclareMathOperator{\Arg}{\mathrm{Arg}}
\DeclareMathOperator{\ra}{\rightarrow}
\DeclareMathOperator{\llra}{\longleftrightarrow}
\DeclareMathOperator{\lra}{\longrightarrow}
\DeclareMathOperator{\dlra}{\Leftrightarrow}
\DeclareMathOperator{\dra}{\Rightarrow}
\newcommand{\gq}{{\ooalign{$ > $\cr\hidewidth $ ? $ \hidewidth\cr}}}

\DeclareMathOperator{\rp}{\mathrm{p}}
\DeclareMathOperator{\rn}{\mathrm{n}}
\DeclareMathOperator{\ru}{\mathrm{u}}
\DeclareMathOperator{\rd}{\mathrm{d}}

\newcommand{\dis}{\displaystyle}
\newcommand{\ft}{\frametitle}
\numberwithin{equation}{section}

\setcounter{tocdepth}{2} % Show sections + subsections

%----------------------------------------------------------------------------------------
%	TITLE PAGE
%----------------------------------------------------------------------------------------

\title[]{An \textsl{Ab Initio} Discussion on\\
	 Anomalous Nuclear Magnetic Moment} % The short title appears at the bottom of every slide, the full title is only on the title page

\author{王石嵘\\161240065} % Your name
\institute[] % Your institution as it will appear on the bottom of every slide, may be shorthand to save space
{Kuang Yaming Honors School\\
%Nanjing University \\ % Your institution for the title page
\medskip
\textit{} % Your email address
}
\date{\today} % Date, can be changed to a custom date

\begin{document}

\begin{frame}
\titlepage % Print the title page as the first slide
\end{frame}

\begin{frame}
\frametitle{Overview} % Table of contents slide, comment this block out to remove it
\tableofcontents % Throughout your presentation, if you choose to use \section{} and \subsection{} commands, these will automatically be printed on this slide as an overview of your presentation
\end{frame}

%----------------------------------------------------------------------------------------
%	PRESENTATION SLIDES
%----------------------------------------------------------------------------------------

\section{Introduction}
\begin{frame}
In Chapter 14 of our textbook, a brief introduction of NMR theory is given.\\
Equation 14.8 gives a na\"ive magnetic moment
\begin{equation}\label{key}
\bm\mu = \mu_N\vb{I} = \dfrac{q}{2m_N}\vb{I}
\end{equation}
and Equation 14.9 modifies that by introducing $ g $-factor
\begin{equation}\label{key}
\bm\mu = g\mu_N\vb{I}
\end{equation}
or 
\begin{equation}\label{key}
\gamma = g\mu_N
\end{equation}
However, why should we introduce $ g $-factor, and can it be explained physically or calculated \textsl{ab initio}?

\end{frame}

\begin{frame}
Nuclear Magnetic Moment, or $ g $-factor, can be measured by experiment, and won't vary in different chemical environments.\\
~\\
However, \textbf{why} the $ g $-factors look \textbf{"anomalous"}?
\begin{table}
	\centering
	\begin{tabular}{ccccc}
		\hline
		& $ \ce{^1H} $ & $ \ce{^2H} $ & $ \ce{^7Li} $ & $ \ce{^19F} $ \\ \hline
		$ g $ & 5.58 & 0.86 & 2.17 & 5.25 \\
		$ \mu $($ \mu_N $) & 2.79 & 0.86 & 3.25 & 2.63\\
		\hline
	\end{tabular}
\end{table}
\end{frame}

%\section{Correction via Dirac Equation}
%\begin{frame}
%
%\end{frame}


\section{\textsl{Ab Initio} Calculation}
\subsection{Proton/Neutron}

\begin{frame}
\frametitle{Calculation of Proton/Neutron Magnetic Moment\cite{particle}}
The magnetic moment of the proton differs from
that expected for a point-like Dirac fermion.\\
Since quarks are fundamental Dirac fermions, the operators for the total magnetic moment and z-component of the magnetic moment are
\begin{align}
\bm\mu = \dfrac{Q}{m}\vb{S} &  & \bm\mu_z = \dfrac{Q}{m}\vb{S}_z
\end{align}
thus
\begin{align}
\mu_{u,z} &= \dfrac{2}{3}\dfrac{1}{m_u}\dfrac{1}{2} = \dfrac{1}{3m_u} =  \dfrac{2m_p}{3m_u}\mu_N \\
\mu_{d,z} &= -\dfrac{1}{3}\dfrac{1}{m_d}\dfrac{1}{2} = -\dfrac{1}{6m_d} =  -\dfrac{m_p}{3m_d}\mu_N
\end{align}
where $ \mu_N = \dfrac{1}{2m_p} $

\end{frame}

\begin{frame}
The proton wavefunction is
\begin{equation}\label{key}
\ket{\rp\uparrow} = \dfrac{1}{\sqrt{6}}(2\ru\uparrow\ru\uparrow\rd\downarrow - \ru\uparrow\ru\downarrow\rd\uparrow - \ru\downarrow\ru\uparrow\rd\uparrow)
\end{equation}
thus
\begin{align}
\mu_p &= \dfrac{1}{6}\Braket{\rp\uparrow | \hat\mu_z^{(1)} + \hat\mu_z^{(2)} + \hat\mu_z^{(3)} | \rp\uparrow} \notag\\
&= \dfrac{4}{6}\Braket{\ru\uparrow\ru\uparrow\rd\downarrow | \hat\mu_z^{(1)} + \hat\mu_z^{(2)} + \hat\mu_z^{(3)} | \ru\uparrow\ru\uparrow\rd\downarrow} \notag\\
&\quad {} + \dfrac{1}{6}\Braket{\ru\uparrow\ru\downarrow\rd\uparrow | \hat\mu_z^{(1)} + \hat\mu_z^{(2)} + \hat\mu_z^{(3)} | \ru\uparrow\ru\downarrow\rd\uparrow} \notag\\
&\quad {} + \dfrac{1}{6}\Braket{\ru\downarrow\ru\uparrow\rd\uparrow | \hat\mu_z^{(1)} + \hat\mu_z^{(2)} + \hat\mu_z^{(3)} | \ru\downarrow\ru\uparrow\rd\uparrow} \notag\\
&= \dfrac{2}{3}(\mu_{\ru} + \mu_{\ru} - \mu_{\rd}) + \dfrac{1}{6}(\mu_{\ru} - \mu_{\ru} + \mu_{\rd}) + \dfrac{1}{6}(-\mu_{\ru} + \mu_{\ru} + \mu_{\rd}) \notag\\
&= \dfrac{4}{3}\mu_{\ru} - \dfrac{1}{3}\mu_{\rd}
\end{align}
thus
\begin{align}
\mu_{\rp} &= \dfrac{4}{3}\dfrac{2m_p}{3m_u}\mu_N - \dfrac{1}{3}\qty(-\dfrac{m_p}{3m_d}\mu_N) = \qty(\dfrac{8}{9}\dfrac{m_p}{m_u} + \dfrac{1}{9}\dfrac{m_p}{m_d})\mu_N
\end{align}

\end{frame}

%\begin{frame}
%Since
%\begin{align}\label{key}
%m_{\rp} = \SI{0.938272}{GeV} & & m_{\ru} = \SI{0.338}{GeV} & & m_{\rd} = %\SI{0.322}{GeV}
%\end{align}
%we have
%\begin{equation}\label{key}
%\mu_{\rp} = \qty(\dfrac{8}{9}\cross 2.775 + \dfrac{1}{9}\cross 2.913)\mu_N = 2.79\mu_N
%\end{equation}
%thus
%\begin{equation}\label{key}
%g_{\rp} = 5.58
%\end{equation}
%\end{frame}

\begin{frame}
Similarly, for neutron %(mass = $ \SI{0.939565}{GeV} $)
\begin{align}
\mu_{\rn}&= \dfrac{4}{3}\mu_{\rd} - \dfrac{1}{3}\mu_{\ru} \notag\\
&= -\qty(\dfrac{4}{9}\dfrac{m_p}{m_d} + \dfrac{2}{9}\dfrac{m_p}{m_u})\mu_N %\notag\\
%&= -\qty(\dfrac{4}{9}\cross 2.913 + \dfrac{2}{9}\cross 2.775)\mu_N \notag\\
%&= 1.911\mu_N
\end{align}
Sadly, we cannot measure mass of quarks directly, due to "color confinement".\\
More sadly, mass of quarks fitted from experiments varies with cases.\\
By na\"ive estimation, we take $ m_{\ru} \approx m_{\rd} \approx \dfrac{1}{3}m_{\rp} \approx \dfrac{1}{3}m_{\rn} $, thus
\begin{align}\label{key}
\mu_{\rp} = 3\mu_N %&& g_{\rp} = 6\\
&& \mu_{\rn} = -2\mu_N %&& g_{\rn} = 4
\end{align}
compared with experimental values
\begin{align}
\mu_{\rp} = 2.79\mu_N && \mu_{\rn} = -1.91\mu_N
\end{align}
The ratio between them is more accurate
\begin{align}
\qty(\dfrac{\mu_{\rp}}{\mu_{\rn}})^{(th)} = -1.5 && \qty(\dfrac{\mu_{\rp}}{\mu_{\rn}})^{(exp)} = -1.46
\end{align}

\end{frame}

\subsection{Multi-baryon nucleus}
\begin{frame}
\frametitle{What about Multi-baryon nucleus?}
classification of nucleus by number of protons and neutrons:
\begin{itemize}
	\item odd-odd
	\item odd-even
	\item even-even
\end{itemize}
Discussion based on Shell Model\cite{nuc}
\begin{table}
	\centering
	\begin{tabular}{ccc}
		\hline
		& spin & magnetic moment\\ \hline
		even-even & 0 & 0\\
		odd-even & by last nucleon & by last nucleon\\
		odd-odd & by coupling of last 2 nucleons & by coupling of last 2 nucleons\\ 
		\hline
	\end{tabular}
\end{table}

\end{frame}


\begin{frame}
References
\printbibliography

\end{frame}

\begin{frame}
\centering
{\Large \textbf{Thank You}}
\end{frame}
%\section{References}

%\bibliography{pre1}
%\bibliographystyle{plain}


\end{document} 