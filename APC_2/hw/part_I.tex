%\documentclass[UTF8]{ctexart} % use larger type; default would be 10pt
\documentclass[a4paper]{article}
\usepackage{xeCJK}
%\usepackage[utf8]{inputenc} % set input encoding (not needed with XeLaTeX)

%%% Examples of Article customizations
% These packages are optional, depending whether you want the features they provide.
% See the LaTeX Companion or other references for full information.

%%% PAGE DIMENSIONS
\usepackage{geometry} % to change the page dimensions
\geometry{a4paper} % or letterpaper (US) or a5paper or....
\geometry{margin=1in} % for example, change the margins to 2 inches all round
% \geometry{landscape} % set up the page for landscape
%   read geometry.pdf for detailed page layout information

\usepackage{graphicx} % support the \includegraphics command and options

% \usepackage[parfill]{parskip} % Activate to begin paragraphs with an empty line rather than an indent

%%% PACKAGES
\usepackage{booktabs} % for much better looking tables
\usepackage{array} % for better arrays (eg matrices) in maths
\usepackage{paralist} % very flexible & customisable lists (eg. enumerate/itemize, etc.)
\usepackage{verbatim} % adds environment for commenting out blocks of text & for better verbatim
\usepackage{subfig} % make it possible to include more than one captioned figure/table in a single float
% These packages are all incorporated in the memoir class to one degree or another...

%%% HEADERS & FOOTERS
\usepackage{fancyhdr} % This should be set AFTER setting up the page geometry
\pagestyle{fancy} % options: empty , plain , fancy
\renewcommand{\headrulewidth}{0pt} % customise the layout...
\lhead{}\chead{}\rhead{}
\lfoot{}\cfoot{\thepage}\rfoot{}

%%% SECTION TITLE APPEARANCE
\usepackage{sectsty}
\allsectionsfont{\sffamily\mdseries\upshape} % (See the fntguide.pdf for font help)
% (This matches ConTeXt defaults)

%%% ToC (table of contents) APPEARANCE
\usepackage[nottoc,notlof,notlot]{tocbibind} % Put the bibliography in the ToC
\usepackage[titles,subfigure]{tocloft} % Alter the style of the Table of Contents
\renewcommand{\cftsecfont}{\rmfamily\mdseries\upshape}
\renewcommand{\cftsecpagefont}{\rmfamily\mdseries\upshape} % No bold!

%%% END Article customizations

%%% The "real" document content comes below...

\setlength{\parindent}{0pt}
\usepackage{physics}
\usepackage{amsmath}
%\usepackage{symbols}
\usepackage{AMSFonts}
\usepackage{bm}
%\usepackage{eucal}
\usepackage{mathrsfs}
\usepackage{amssymb}
\usepackage{float}
\usepackage{multicol}
\usepackage{abstract}
\usepackage{empheq}
\usepackage{extarrows}
\usepackage{textcomp}
\usepackage{fontspec}
\usepackage{braket}
\usepackage{siunitx}
\sisetup{
	separate-uncertainty = true,
	inter-unit-product = \ensuremath{{}\cdot{}}
}
\usepackage{mhchem}
\usepackage{hyperref}
\hypersetup{
	colorlinks=true,
	linkcolor=black,
	filecolor=magenta,      
	urlcolor=cyan,
}

\DeclareMathOperator{\p}{\prime}
\DeclareMathOperator{\ti}{\times}
\DeclareMathOperator{\intinf}{\int_0^\infty}
\DeclareMathOperator{\intdinf}{\int_{-\infty}^\infty}
\DeclareMathOperator{\intzpi}{\int_0^\pi}
\DeclareMathOperator{\intztpi}{\int_0^{2\pi}}
\DeclareMathOperator{\sumninf}{\sum_{n=1}^{\infty}}
\DeclareMathOperator{\sumninfz}{\sum_{n=0}^\infty}
\DeclareMathOperator{\sumiinf}{\sum_{i=1}^{\infty}}
\DeclareMathOperator{\sumiinfz}{\sum_{i=0}^\infty}
\DeclareMathOperator{\sumkinf}{\sum_{k=1}^{\infty}}
\DeclareMathOperator{\sumkinfz}{\sum_{k=0}^\infty}
\DeclareMathOperator{\e}{\mathrm{e}}
\DeclareMathOperator{\I}{\mathrm{i}}
\DeclareMathOperator{\Arg}{\mathrm{Arg}}
\newcommand{\NA}{N_\mathrm{A}}
\newcommand{\kB}{k_\mathrm{B}}

\DeclareMathOperator{\ra}{\rightarrow}
\DeclareMathOperator{\llra}{\longleftrightarrow}
\DeclareMathOperator{\lra}{\longrightarrow}
\DeclareMathOperator{\dlra}{\Leftrightarrow}
\DeclareMathOperator{\dra}{\Rightarrow}
\newcommand{\bkk}[1]{\Braket{#1|#1}}
\newcommand{\bk}[2]{\Braket{#1|#2}}
\newcommand{\bkev}[2]{\Braket{#2|#1|#2}}



\DeclareMathOperator{\hV}{\hat{\vb{V}}}

\DeclareMathOperator{\hx}{\hat{\vb{x}}}
\DeclareMathOperator{\hy}{\hat{\vb{y}}}
\DeclareMathOperator{\hz}{\hat{\vb{z}}}

\DeclareMathOperator{\hA}{\hat{\vb{A}}}

\DeclareMathOperator{\hQ}{\hat{\vb{Q}}}
\DeclareMathOperator{\hI}{\hat{\vb{I}}}
\DeclareMathOperator{\psis}{\psi^\ast}
\DeclareMathOperator{\Psis}{\Psi^\ast}
\DeclareMathOperator{\hi}{\hat{\vb{i}}}
\DeclareMathOperator{\hj}{\hat{\vb{j}}}
\DeclareMathOperator{\hk}{\hat{\vb{k}}}
\DeclareMathOperator{\hr}{\hat{\vb{r}}}
\DeclareMathOperator{\hT}{\hat{\vb{T}}}
\DeclareMathOperator{\hH}{\hat{H}}
\DeclareMathOperator{\hh}{\hat{h}}               % helicity
\DeclareMathOperator{\hL}{\hat{\vb{L}}}
\DeclareMathOperator{\hp}{\hat{\vb{p}}}

\DeclareMathOperator{\ha}{\hat{\vb{a}}}
\DeclareMathOperator{\hS}{\hat{\vb{S}}}
\DeclareMathOperator{\hSigma}{\hat{\bm\Sigma}}
\DeclareMathOperator{\hJ}{\hat{\vb{J}}}
\DeclareMathOperator{\hP}{\hat{\vb{P}}}          % Parity
\DeclareMathOperator{\hC}{\hat{\vb{C}}} 
\DeclareMathOperator{\Tdv}{-\dfrac{\hbar^2}{2m}\dv[2]{x}}
\DeclareMathOperator{\Tna}{-\dfrac{\hbar^2}{2m}\nabla^2}
\DeclareMathOperator{\vna}{\vnabla}
\DeclareMathOperator{\nna}{\nabla^2}
\newcommand{\naCarExpd}[1]{\pdv[2]{#1}{x} + \pdv[2]{#1}{y} + \pdv[2]{#1}{z}}
\newcommand{\naCyl}{\qty[\dfrac{1}{\rho}\pdv{\rho}\qty(\rho\pdv{\rho}) + \dfrac{1}{\rho^2}\pdv[2]{\phi} + \pdv[2]{z}]}

%\DeclareMathOperator{\g#0}{\gamma^0}
%\DeclareMathOperator{\g1}{\gamma^1}
%\DeclareMathOperator{\g2}{\gamma^2}
%\DeclareMathOperator{\g3}{\gamma^3}
%\DeclareMathOperator{\g5}{\gamma^5}
\newcommand{\g}[1]{\gamma^{#1}}
\DeclareMathOperator{\gmuu}{\gamma^\mu}
\DeclareMathOperator{\gmud}{\gamma_\mu}
%\newcommand{\G}[2]{g^{#1#2}}

\newcommand{\subsbul}{\subsection*{$ \bullet $}}
\newcommand{\ex}[1]{\paragraph{#1}}
\newcommand{\subex}[1]{\subparagraph{#1}}
\newcommand{\dis}{\displaystyle}
\newcommand{\iden}{{\large \bm{1}}}
\newcommand{\qed}{$ \Square $}
\newcommand{\tPhi}{\tilde{\Phi} }
\DeclareMathOperator{\au}{\mathrm{a.u.}}
\newcommand{\ntg}{\notag\\}
\DeclareMathOperator{\rms}{\mathrm{rms}}

\numberwithin{equation}{section}
%\setcounter{secnumdepth}{4}
\setcounter{tocdepth}{4}
\allowdisplaybreaks[4]

\usepackage{xcolor}
\definecolor{codegray}{gray}{0.9}
\newfontfamily\Consolas{Consolas}
\newcommand{\code}[1]{\colorbox{codegray}{{\Consolas#1}}}

\title{\textbf{Advanced Physical Chemistry II}\\HW   Part I}
\author{王石嵘
\vspace{5pt}\\
161240065\\
%Email: shirong\_wang@berkeley.edu
}
\date{\today} % Activate to display a given date or no date (if empty),
         % otherwise the current date is printed 

\begin{document}
% \boldmath

\maketitle

%\tableofcontents

%\newpage

\setcounter{section}{24}
\section{The Kinetic Theory of Gases}
2,3,17,26,27,35,37,42\\
\ex{25-2}
\begin{equation}\label{key}
u_{\rms} = \sqrt{\dfrac{3RT}{M}} = \sqrt{\dfrac{3 R T}{\num{28.02e-3}}}
\end{equation}
thus
\begin{align}
u_{\rms}(\SI{200}{K}) &= \SI{421.95}{m/s} \\
u_{\rms}(\SI{300}{K}) &= \SI{516.78}{m/s} \\
u_{\rms}(\SI{500}{K}) &= \SI{667.16}{m/s} \\
u_{\rms}(\SI{1000}{K}) &= \SI{943.50}{m/s} 
\end{align}

\ex{25-3}
Since
\begin{equation}\label{key}
u_{\rms} = \sqrt{\dfrac{3RT}{M}}
\end{equation}
The RMS speed is increased by $ \sqrt{2} $.


\ex{25-17}
Since
\begin{equation}\label{key}
f(u_x) = \sqrt{\dfrac{m}{2\pi\kB T}}\e^{-mu_x^2/2\kB T}
\end{equation}
when $ u_x>0 $
\begin{align}
\ev{u_x} &= \intinf u_x f(u_x)\dd u_x = \sqrt{\dfrac{m}{2\pi\kB T}} \intinf u_x\e^{-mu_x^2/2\kB T} \dd u_x \notag\\
&= \sqrt{\dfrac{m}{2\pi\kB T}} \qty(-\dfrac{\kB T}{m})(0-1) \notag\\
&= \sqrt{\dfrac{\kB T}{2\pi m}}
\end{align}

\ex{25-26}
Since
\begin{equation}\label{key}
F(\varepsilon) = \dfrac{2\pi}{(\pi \kB T)^{3/2}}\varepsilon^{1/2}\e^{-\varepsilon/\kB T}
\end{equation}
Let $ \dv{F}{\varepsilon} = 0 $, we have
\begin{equation}\label{key}
\dfrac{1}{2}\varepsilon^{-1/2}\e^{-\varepsilon/\kB T} + \varepsilon^{1/2}(-\dfrac{1}{\kB T})\e^{-\varepsilon/\kB T} = 0
\end{equation}
\begin{equation}\label{key}
\varepsilon = \dfrac{\kB T}{2}
\end{equation}

\ex{25-27}
\begin{align}
\ev{\varepsilon} &= \intinf \varepsilon F(\varepsilon)\dd\varepsilon \notag\\
&= \intinf \dfrac{2\pi}{(\pi \kB T)^{3/2}} \varepsilon^{3/2}\e^{-\varepsilon/\kB T}\dd\varepsilon \notag\\
&= \dfrac{2\pi}{(\pi \kB T)^{3/2}} \dfrac{3}{4}(\kB T)^{5/2}\sqrt{\pi} \notag\\
&= \dfrac{3}{2}\kB T
\end{align}
\begin{align}
\ev{\varepsilon^2} &= \intinf \varepsilon^2 F(\varepsilon)\dd\varepsilon \notag\\
&= \intinf \dfrac{2\pi}{(\pi \kB T)^{3/2}} \varepsilon^{5/2}\e^{-\varepsilon/\kB T}\dd\varepsilon \notag\\
&= \dfrac{2\pi}{(\pi \kB T)^{3/2}} \dfrac{15}{8}(\kB T)^{7/2}\sqrt{\pi} \notag\\
&=  \dfrac{15}{4}(\kB T)^2
\end{align}
\begin{equation}\label{key}
\sigma_\varepsilon^2 = \ev{\varepsilon^2} - \ev{\varepsilon}^2 = \dfrac{3}{2}(\kB T)^2
\end{equation}
thus
\begin{equation}\label{key}
\dfrac{\sigma_\varepsilon}{\ev{\varepsilon}} = \sqrt{\dfrac{3}{2}}{\Big /}\dfrac{3}{2} = \sqrt{\dfrac{2}{3}}
\end{equation}
which means the fluctuations in $ \varepsilon $ are large with respect to $ \varepsilon $.

\ex{25-35}
\begin{align}\label{key}
z_A &= \rho\sigma\sqrt{2}\sqrt{\dfrac{8RT}{\pi M}} = \dfrac{P \NA}{R T}\sigma \cdot 4\sqrt{\dfrac{RT}{\pi M}} \notag\\
&= \dfrac{4\sigma \NA}{\sqrt{\pi M RT}}P 
\end{align}
where $ \sigma = \SI{0.230e-18}{m^2} $
\subex{(a)}
\begin{equation}\label{key}
z_A = \dfrac{4\times\num{0.230e-18} \times \num{6.022e23}}{\sqrt{\pi\times \num{2.016e-3} \times\num{8.3145}\times 298.15}}\times 133.32 = \SI{1.86e7}{s^{-1}}
\end{equation}
\subex{(b)}
\begin{equation}\label{key}
z_A = \dfrac{4\times\num{0.230e-18} \times \num{6.022e23}}{\sqrt{\pi\times \num{2.016e-3} \times\num{8.3145}\times 298.15}}\times \num{1e5} = \SI{1.40e10}{Hz}
\end{equation}


\ex{25-37}
The probability that an $ \ce{O_2} $ molecule will travel distance $ d $ without a collision is
\begin{align}\label{key}
P(d) &= 1 - \int_0^d p(x)\dd x = 1 - \int_0^d \dfrac{1}{l}\e^{-x/l}\dd x \notag\\
&= 1 - \dfrac{1}{l}(-l\e^{-x/l})\Big|_0^d \notag\\
&= 1 + (\e^{-d/l} - 1) \notag\\
&= \e^{-d/l}
\end{align}
Since the MFP
\begin{align}\label{key}
l &= \dfrac{1}{\sqrt{2}\rho\sigma} = \dfrac{\kB T}{\sqrt{2}\sigma P} \notag\\
&= \dfrac{\num{1.38e-23}\times 298.15}{\sqrt{2}\times\num{0.410e-18}\times\num{1e5}} \notag\\
&= \SI{7.10e-8}{m} = \SI{7.10e-5}{mm}
\end{align}
we get
\subex{(a)}
\begin{equation}\label{key}
P(\SI{1.00e-5}{mm}) = \e^{-\num{1.00e-5}/\num{7.10e-5}} = 0.869
\end{equation}
\subex{(b)}
\begin{equation}\label{key}
P(\SI{1.00e-3}{mm}) = \e^{-\num{1.00e-3}/\num{7.10e-5}} = \num{7.63e-7}
\end{equation}
\subex{(c)}
\begin{equation}\label{key}
P(\SI{1.00}{mm}) = \e^{-\num{1.00}/\num{7.10e-5}} = \num{4.20e-6118}
\end{equation}

\ex{25-42}
Since 
\begin{equation}\label{key}
l = \dfrac{1}{\sqrt{2}\rho\sigma} = \dfrac{\kB T}{\sqrt{2}\sigma P} 
\end{equation}
we have
\begin{equation}\label{key}
P(l) = \dfrac{\kB T}{\sqrt{2}\sigma l}
\end{equation}
where $ \sigma=\SI{0.230e-18}{m^2} $, $ T=\SI{293.15}{K} $.\\
thus
\begin{align}
P(\SI{100}{\mu m}) &= \SI{124}{Pa} \notag\\
P(\SI{1.00}{mm}) &= \SI{12.4}{Pa} \notag\\
P(\SI{1.00}{m}) &= \SI{0.0124}{Pa} 
\end{align}

\section{Chemical Kinetics I: Rate Laws}
\ex{26-47}
\begin{align}
\Delta^\ddagger G^\circ &= \Delta^\ddagger H^\circ - T\Delta^\ddagger S^\circ \notag\\
&= 31.38 - 325\times \num{16.74e-3} \notag\\
&= \SI{25.94}{kJ/mol}
\end{align}
\begin{align}
k &= \dfrac{\kB T}{h c^\circ}\e^{-\Delta^\ddagger G^\circ/RT} \notag\\
&= \dfrac{\num{1.38e-23}\times 325}{6.626e-34\times 1} \e^{\num{-25.94e3}/8.314\times 325} \notag\\
&= \SI{4.59}{s^{-1}}
\end{align}


\setcounter{section}{27}
\section{The Rate of a Bimolecular Gas-Phase Reaction}
1,4,6,10\\
\ex{28-1}
%Using collision diameters in Table 25.3 of textbook, we have
The cross section between $ \ce{NO} $ and $ \ce{Cl_2} $
\begin{equation}\label{key}
\sigma = \pi d^2 = \pi \qty(\dfrac{370 + 540}{2}) = \SI{6.50e5}{pm^2} = \SI{6.50e-19}{m^2}
\end{equation}
the reduced mass
\begin{equation}\label{key}
\mu = \dfrac{m_{\ce{NO}}\times m_{\ce{Cl_2}}}{m_{\ce{NO}} + m_{\ce{Cl_2}}} = \dfrac{70.906\times 30.006}{70.906 + 30.006} = \SI{21.0838}{amu} = \SI{3.5010e-26}{kg}
\end{equation}
thus 
\begin{equation}\label{key}
\ev{u_r} = \sqrt{\dfrac{8\kB T}{\pi \mu}} = \sqrt{\dfrac{8\times \kB\times 300}{\pi\times \num{3.5010e-26}}} = \SI{548.88}{m/s}
\end{equation}


\ex{28-4}


\ex{28-6}



\ex{28-10}



Additional Problems\\
\ex{1.} 对于单分子气相反应,活化熵变往往可忽略不计,试计算按室温(\SI{200}{K})附近活化焓分别为60,80,100$ \si{kJ.mol^{-1}} $时之反应比速及$ t_{1/2} $。


\ex{2}



\end{document}