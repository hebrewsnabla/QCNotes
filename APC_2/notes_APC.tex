%\documentclass[UTF8]{ctexart} % use larger type; default would be 10pt
\documentclass[a4paper]{article}
\usepackage{xeCJK}
%\usepackage{ctex}
%\usepackage{luatexja-fontspec}
%\setmainjfont{FandolSong}
%\usepackage[utf8]{inputenc} % set input encoding (not needed with XeLaTeX)

%%% Examples of Article customizations
% These packages are optional, depending whether you want the features they provide.
% See the LaTeX Companion or other references for full information.

%%% PAGE DIMENSIONS
\usepackage{geometry} % to change the page dimensions
\geometry{a4paper} % or letterpaper (US) or a5paper or....
% \geometry{margin=2in} % for example, change the margins to 2 inches all round
% \geometry{landscape} % set up the page for landscape
%   read geometry.pdf for detailed page layout information

\usepackage{graphicx} % support the \includegraphics command and options

% \usepackage[parfill]{parskip} % Activate to begin paragraphs with an empty line rather than an indent

%%% PACKAGES
\usepackage{booktabs} % for much better looking tables
\usepackage{array} % for better arrays (eg matrices) in maths
\usepackage{paralist} % very flexible & customisable lists (eg. enumerate/itemize, etc.)
\usepackage{verbatim} % adds environment for commenting out blocks of text & for better verbatim
\usepackage{subfig} % make it possible to include more than one captioned figure/table in a single float
% These packages are all incorporated in the memoir class to one degree or another...

%%% HEADERS & FOOTERS
\usepackage{fancyhdr} % This should be set AFTER setting up the page geometry
\pagestyle{fancy} % options: empty , plain , fancy
\renewcommand{\headrulewidth}{0pt} % customise the layout...
\lhead{}\chead{}\rhead{}
\lfoot{}\cfoot{\thepage}\rfoot{}

%%% SECTION TITLE APPEARANCE
\usepackage{sectsty}
\allsectionsfont{\sffamily\mdseries\upshape} % (See the fntguide.pdf for font help)
% (This matches ConTeXt defaults)

%%% ToC (table of contents) APPEARANCE
\usepackage[nottoc,notlof,notlot]{tocbibind} % Put the bibliography in the ToC
\usepackage[titles,subfigure]{tocloft} % Alter the style of the Table of Contents
\renewcommand{\cftsecfont}{\rmfamily\mdseries\upshape}
\renewcommand{\cftsecpagefont}{\rmfamily\mdseries\upshape} % No bold!

%%% END Article customizations

%%% The "real" document content comes below...

\setlength{\parindent}{0pt}
\usepackage{physics}
\usepackage{amsmath}
%\usepackage{symbols}
\usepackage{AMSFonts}
\usepackage{bm}
%\usepackage{eucal}
\usepackage{mathrsfs}
\usepackage{amssymb}
\usepackage{float}
\usepackage{multicol}
\usepackage{abstract}
\usepackage{empheq}
\usepackage{extarrows}
\usepackage{textcomp}
\usepackage{mhchem}
\usepackage{braket}
\usepackage{siunitx}
\usepackage[utf8]{inputenc}
\usepackage{tikz-feynman}
\usepackage{feynmp}


\DeclareMathOperator{\p}{\prime}
\DeclareMathOperator{\ti}{\times}

\DeclareMathOperator{\e}{\mathrm{e}}
\DeclareMathOperator{\I}{\mathrm{i}}
\DeclareMathOperator{\Arg}{\mathrm{Arg}}
\newcommand{\NA}{N_\mathrm{A}}
\newcommand{\kB}{k_\mathrm{B}}

\DeclareMathOperator{\ra}{\rightarrow}
\DeclareMathOperator{\llra}{\longleftrightarrow}
\DeclareMathOperator{\lra}{\longrightarrow}
\DeclareMathOperator{\dlra}{\;\Leftrightarrow\;}
\DeclareMathOperator{\dra}{\;\Rightarrow\;}

%%%%%%%%%%%% QUANTUM MECHANICS %%%%%%%%%%%%%%%%%%%%%%%%
\newcommand{\bkk}[1]{\Braket{#1|#1}}
\newcommand{\bk}[2]{\Braket{#1|#2}}
\newcommand{\bkev}[2]{\Braket{#2|#1|#2}}

\DeclareMathOperator{\na}{\bm{\nabla}}
\DeclareMathOperator{\nna}{\nabla^2}
\DeclareMathOperator{\drrr}{\dd[3]\vb{r}}

\DeclareMathOperator{\psis}{\psi^\ast}
\DeclareMathOperator{\Psis}{\Psi^\ast}
\DeclareMathOperator{\hi}{\hat{\vb{i}}}
\DeclareMathOperator{\hj}{\hat{\vb{j}}}
\DeclareMathOperator{\hk}{\hat{\vb{k}}}
\DeclareMathOperator{\hr}{\hat{\vb{r}}}
\DeclareMathOperator{\hT}{\hat{\vb{T}}}
\DeclareMathOperator{\hH}{\hat{H}}

\DeclareMathOperator{\hL}{\hat{\vb{L}}}
\DeclareMathOperator{\hp}{\hat{\vb{p}}}
\DeclareMathOperator{\hx}{\hat{\vb{x}}}
\DeclareMathOperator{\ha}{\hat{\vb{a}}}
\DeclareMathOperator{\hS}{\hat{\vb{S}}}
\DeclareMathOperator{\hSigma}{\hat{\bm\Sigma}}
\DeclareMathOperator{\hJ}{\hat{\vb{J}}}

\DeclareMathOperator{\Tdv}{-\dfrac{\hbar^2}{2m}\dv[2]{x}}
\DeclareMathOperator{\Tna}{-\dfrac{\hbar^2}{2m}\nabla^2}

%\DeclareMathOperator{\s}{\sum_{n=1}^{\infty}}
\DeclareMathOperator{\intinf}{\int_0^\infty}
\DeclareMathOperator{\intdinf}{\int_{-\infty}^\infty}
%\DeclareMathOperator{\suminf}{\sum_{n=0}^\infty}
\DeclareMathOperator{\sumnzinf}{\sum_{n=0}^\infty}
\DeclareMathOperator{\sumnoinf}{\sum_{n=1}^\infty}
\DeclareMathOperator{\sumndinf}{\sum_{n=-\infty}^\infty}
\DeclareMathOperator{\sumizinf}{\sum_{i=0}^\infty}

%%%%%%%%%%%%%%%%% PARTICLE PHYSICS %%%%%%%%%%%%%%%%
\DeclareMathOperator{\hh}{\hat{h}}               % helicity
\DeclareMathOperator{\hP}{\hat{\vb{P}}}          % Parity
\DeclareMathOperator{\hU}{\hat{U}}
\DeclareMathOperator{\hG}{\hat{G}}

\DeclareMathOperator{\GeV}{\si{GeV}}
\DeclareMathOperator{\LI}{\mathscr{L}.I.}
%\DeclareMathOperator{\g5}{\gamma^5}
\DeclareMathOperator{\gmuu}{\gamma^\mu}
\DeclareMathOperator{\gmud}{\gamma_\mu}
\DeclareMathOperator{\gnuu}{\gamma^\nu}
\DeclareMathOperator{\gnud}{\gamma_\nu}

\renewcommand{\u}{\mathrm{u}}
\renewcommand{\d}{\mathrm{d}}
\DeclareMathOperator{\s}{\mathrm{s}}

\DeclareMathOperator{\q}{\mathrm{q}}
\DeclareMathOperator{\bq}{\bar{\mathrm{q}}}

\DeclareMathOperator{\g}{\mathrm{g}}
\DeclareMathOperator{\W}{\mathrm{W}}
\DeclareMathOperator{\Z}{\mathrm{Z}}

%%% Feynman Diagram
\newcommand{\pa}{particle}
\newcommand{\mo}{momentum}
\newcommand{\el}{edge label}

\newcommand{\dis}{\displaystyle}
\numberwithin{equation}{section}
\setcounter{tocdepth}{4}

\title{Notes of \textbf{Advanced Physical Chemistry II}}
\author{hebrewsnabla}
%\date{} % Activate to display a given date or no date (if empty),
         % otherwise the current date is printed 

\begin{document}
% \boldmath
\maketitle

\tableofcontents

\newpage


\section*{Introduction}
TA: 刘琼 G403

\setcounter{section}{11}
\section{Group Theory: the Exploitation of Symmetry}
\subsection*{Matrices}
$ \det(\vb{A}) = 0 \;\dra\; \vb{A} $ is a singular matrix.\\

\subsection{The Exploitation of the Symm of a Mol Can Be Used to Significantly Simplify Numerical Calculations}

\subsection{The Symm of Mols Can Be Described by a Set of Symm Elements}
\begin{table}[H]
	\centering
	\begin{tabular}{cccc}
		\hline
		$ E $ & & & \\
		$ C_n $ & & & Rotation by $ 360\textdegree/n $\\
		$ \sigma $ && &\\
		$ i $ &&&\\
		$ S_n $ &&&\\
		\hline
	\end{tabular}
    \caption{Symmetry elements and operators}
\end{table}
\paragraph{Identity}
\paragraph{Rotation}
\paragraph{Reflection}
\begin{table}[H]
	\centering
	\begin{tabular}{cc}
		\hline
		$ \sigma_h $ & horizontal\\
		$ \sigma_v $ & vertical\\
		$ \sigma_d $ & diagonal (vertical and bisects the angle between $ C_2 $ axis)\\
		\hline
	\end{tabular}
	\caption{}
\end{table}
\paragraph{Inversion}
\paragraph{Rotation Reflection}
\begin{equation}\label{key}
\hat{S}_n = \hat\sigma_h \cross \hat{C}_n
\end{equation}

\subsubsection{Point Groups of Interest to Chemists}
\begin{table}[H]
	\centering
	\begin{tabular}{ccc}
		\hline
		$ C_{nv} $ & &  \\
		$ C_{nh} $ &  & Rotation by $ 360\textdegree/n $\\
		$ D_{nh} $ & &\\
		$ D_{nv} $ &&\\
		$ D_{nd} $ &&\\
		$ T_d $ &&\\
		\hline
	\end{tabular}
	\caption{Symmetry elements and operators}
\end{table}

\subsection{The Symm Operators of a Mol Form a Group}
A set of operators form a group if they satisfy:
\begin{enumerate}
	\item closed under multiplication 乘法封闭
	\item associative multiplication 乘法结合律
	\item only one identity operator 单位元
	\item everyone has only one inverse 逆元
\end{enumerate}

\subsubsection{Point Group for Some Mols}
\paragraph{No Symm Axis}~\\
$ C_1 $ -- nothing\\
$ C_s $ -- $ \sigma $\\
$ C_i $ -- $ i $\\
\paragraph{$ C_n $}
\paragraph{$ S_n $}
\paragraph{$ C_{nv} $} -- $ C_n $ and $ n\sigma_v $
\paragraph{$ C_{nh} $} -- $ C_n $ and $ \sigma_h $
\paragraph{$ D_n $} -- $ C_n $ and $ nC_2 \perp C_n$\\
e.g. 一点点交错的$ \ce{C_3H_6} $, $ C_2 $在3个角平分线处
\paragraph{$ D_{nd} $} -- $ C_n $(also $ S_{2n} $) and $ nC_2 \perp C_n$ and $ n\sigma_d $ 
\paragraph{$ D_{nh} $} -- $ C_n $ and $ nC_2 \perp C_n$ and $ \sigma_h $ 
\paragraph{$ T_d $} 主轴是$ S_4 $
\paragraph{$ O_h $}
\paragraph{$ I_h $}

\subsection{Symm Operators Can Be Represented by Matrices}

\subsection{The $ C_{3v} $ Point Group Has a 2-D Irreducible Representation}

\subsection{The Most Important Summary of the Properties of a Point Group Is Its Character Table}
\begin{equation}\label{key}
\sum_R D_{il}^{(\nu)} D_{jm}^{*(\mu)} = \dfrac{g}{n_\nu} \delta_{\mu\nu}\delta_{ij}\delta_{lm}
\end{equation}
\begin{equation}\label{key}
\sum_{R} \chi^{(\nu)}(R)\chi^{*(\mu)}(R) = g\delta_{\mu\nu}
\end{equation}

\end{document}