%\documentclass[UTF8]{ctexart} % use larger type; default would be 10pt
\documentclass[a4paper]{article}
\usepackage{xeCJK}
%\usepackage{ctex}
%\usepackage{luatexja-fontspec}
%\setmainjfont{FandolSong}
%\usepackage[utf8]{inputenc} % set input encoding (not needed with XeLaTeX)

%%% Examples of Article customizations
% These packages are optional, depending whether you want the features they provide.
% See the LaTeX Companion or other references for full information.

%%% PAGE DIMENSIONS
\usepackage{geometry} % to change the page dimensions
\geometry{a4paper} % or letterpaper (US) or a5paper or....
% \geometry{margin=2in} % for example, change the margins to 2 inches all round
% \geometry{landscape} % set up the page for landscape
%   read geometry.pdf for detailed page layout information

\usepackage{graphicx} % support the \includegraphics command and options

% \usepackage[parfill]{parskip} % Activate to begin paragraphs with an empty line rather than an indent

%%% PACKAGES
\usepackage{booktabs} % for much better looking tables
\usepackage{array} % for better arrays (eg matrices) in maths
\usepackage{paralist} % very flexible & customisable lists (eg. enumerate/itemize, etc.)
\usepackage{verbatim} % adds environment for commenting out blocks of text & for better verbatim
\usepackage{subfig} % make it possible to include more than one captioned figure/table in a single float
% These packages are all incorporated in the memoir class to one degree or another...

%%% HEADERS & FOOTERS
\usepackage{fancyhdr} % This should be set AFTER setting up the page geometry
\pagestyle{fancy} % options: empty , plain , fancy
\renewcommand{\headrulewidth}{0pt} % customise the layout...
\lhead{}\chead{}\rhead{}
\lfoot{}\cfoot{\thepage}\rfoot{}

%%% SECTION TITLE APPEARANCE
\usepackage{sectsty}
\allsectionsfont{\sffamily\mdseries\upshape} % (See the fntguide.pdf for font help)
% (This matches ConTeXt defaults)

%%% ToC (table of contents) APPEARANCE
\usepackage[nottoc,notlof,notlot]{tocbibind} % Put the bibliography in the ToC
\usepackage[titles,subfigure]{tocloft} % Alter the style of the Table of Contents
\renewcommand{\cftsecfont}{\rmfamily\mdseries\upshape}
\renewcommand{\cftsecpagefont}{\rmfamily\mdseries\upshape} % No bold!

%%% END Article customizations

%%% The "real" document content comes below...

\setlength{\parindent}{0pt}
\usepackage{physics}
\usepackage{amsmath}
%\usepackage{symbols}
\usepackage{AMSFonts}
\usepackage{bm}
%\usepackage{eucal}
\usepackage{mathrsfs}
\usepackage{amssymb}
\usepackage{float}
\usepackage{multicol}
\usepackage{abstract}
\usepackage{empheq}
\usepackage{extarrows}
\usepackage{textcomp}
\usepackage{mhchem}
\usepackage{braket}
\usepackage{siunitx}
\usepackage[utf8]{inputenc}
\usepackage{tikz-feynman}
\usepackage{feynmp}


\DeclareMathOperator{\p}{\prime}
\DeclareMathOperator{\ti}{\times}

\DeclareMathOperator{\e}{\mathrm{e}}
\DeclareMathOperator{\I}{\mathrm{i}}
\DeclareMathOperator{\Arg}{\mathrm{Arg}}
\newcommand{\NA}{N_\mathrm{A}}
\newcommand{\kB}{k_\mathrm{B}}

\DeclareMathOperator{\ra}{\rightarrow}
\DeclareMathOperator{\llra}{\longleftrightarrow}
\DeclareMathOperator{\lra}{\longrightarrow}
\DeclareMathOperator{\dlra}{\;\Leftrightarrow\;}
\DeclareMathOperator{\dra}{\;\Rightarrow\;}

%%%%%%%%%%%% QUANTUM MECHANICS %%%%%%%%%%%%%%%%%%%%%%%%
\newcommand{\bkk}[1]{\Braket{#1|#1}}
\newcommand{\bk}[2]{\Braket{#1|#2}}
\newcommand{\bkev}[2]{\Braket{#2|#1|#2}}

\DeclareMathOperator{\na}{\bm{\nabla}}
\DeclareMathOperator{\nna}{\nabla^2}
\DeclareMathOperator{\drrr}{\dd[3]\vb{r}}

\DeclareMathOperator{\psis}{\psi^\ast}
\DeclareMathOperator{\Psis}{\Psi^\ast}
\DeclareMathOperator{\hi}{\hat{\vb{i}}}
\DeclareMathOperator{\hj}{\hat{\vb{j}}}
\DeclareMathOperator{\hk}{\hat{\vb{k}}}
\DeclareMathOperator{\hr}{\hat{\vb{r}}}
\DeclareMathOperator{\hT}{\hat{\vb{T}}}
\DeclareMathOperator{\hH}{\hat{H}}

\DeclareMathOperator{\hL}{\hat{\vb{L}}}
\DeclareMathOperator{\hp}{\hat{\vb{p}}}
\DeclareMathOperator{\hx}{\hat{\vb{x}}}
\DeclareMathOperator{\ha}{\hat{\vb{a}}}
\DeclareMathOperator{\hS}{\hat{\vb{S}}}
\DeclareMathOperator{\hSigma}{\hat{\bm\Sigma}}
\DeclareMathOperator{\hJ}{\hat{\vb{J}}}

\DeclareMathOperator{\Tdv}{-\dfrac{\hbar^2}{2m}\dv[2]{x}}
\DeclareMathOperator{\Tna}{-\dfrac{\hbar^2}{2m}\nabla^2}

%\DeclareMathOperator{\s}{\sum_{n=1}^{\infty}}
\DeclareMathOperator{\intinf}{\int_0^\infty}
\DeclareMathOperator{\intdinf}{\int_{-\infty}^\infty}
%\DeclareMathOperator{\suminf}{\sum_{n=0}^\infty}
\DeclareMathOperator{\sumnzinf}{\sum_{n=0}^\infty}
\DeclareMathOperator{\sumnoinf}{\sum_{n=1}^\infty}
\DeclareMathOperator{\sumndinf}{\sum_{n=-\infty}^\infty}
\DeclareMathOperator{\sumizinf}{\sum_{i=0}^\infty}

%%%%%%%%%%%%%%%%% PARTICLE PHYSICS %%%%%%%%%%%%%%%%
\DeclareMathOperator{\hh}{\hat{h}}               % helicity
\DeclareMathOperator{\hP}{\hat{\vb{P}}}          % Parity
\DeclareMathOperator{\hU}{\hat{U}}
\DeclareMathOperator{\hG}{\hat{G}}

\DeclareMathOperator{\GeV}{\si{GeV}}
\DeclareMathOperator{\LI}{\mathscr{L}.I.}
%\DeclareMathOperator{\g5}{\gamma^5}
\DeclareMathOperator{\gmuu}{\gamma^\mu}
\DeclareMathOperator{\gmud}{\gamma_\mu}
\DeclareMathOperator{\gnuu}{\gamma^\nu}
\DeclareMathOperator{\gnud}{\gamma_\nu}

\renewcommand{\u}{\mathrm{u}}
\renewcommand{\d}{\mathrm{d}}
\DeclareMathOperator{\s}{\mathrm{s}}

\DeclareMathOperator{\q}{\mathrm{q}}
\DeclareMathOperator{\bq}{\bar{\mathrm{q}}}

\DeclareMathOperator{\g}{\mathrm{g}}
\DeclareMathOperator{\W}{\mathrm{W}}
\DeclareMathOperator{\Z}{\mathrm{Z}}

%%% Feynman Diagram
\newcommand{\pa}{particle}
\newcommand{\mo}{momentum}
\newcommand{\el}{edge label}

%%% APC
\DeclareMathOperator{\hR}{\hat{\vb{R}}}
\DeclareMathOperator{\trans}{\mathrm{trans}}
\DeclareMathOperator{\rot}{\mathrm{rot}}
\DeclareMathOperator{\vib}{\mathrm{vib}}
\DeclareMathOperator{\elec}{\mathrm{elec}}

\newcommand{\dis}{\displaystyle}
\numberwithin{equation}{section}
\setcounter{tocdepth}{4}

\title{Notes of \textbf{Advanced Physical Chemistry II}}
\author{hebrewsnabla}
%\date{} % Activate to display a given date or no date (if empty),
         % otherwise the current date is printed 

\begin{document}
% \boldmath
\maketitle

\tableofcontents

\newpage


\section*{Introduction}
TA: 刘琼 G403

\setcounter{section}{11}
\section{Group Theory: the Exploitation of Symmetry}
\subsection*{Matrices}
$ \det(\vb{A}) = 0 \;\dra\; \vb{A} $ is a singular matrix.\\

\subsection{The Exploitation of the Symm of a Mol Can Be Used to Significantly Simplify Numerical Calculations}

\subsection{The Symm of Mols Can Be Described by a Set of Symm Elements}
\begin{table}[H]
	\centering
	\begin{tabular}{cccc}
		\hline
		$ E $ & & & \\
		$ C_n $ & & & Rotation by $ 360\textdegree/n $\\
		$ \sigma $ && &\\
		$ i $ &&&\\
		$ S_n $ &&&\\
		\hline
	\end{tabular}
    \caption{Symmetry elements and operators}
\end{table}
\paragraph{Identity}
\paragraph{Rotation}
\paragraph{Reflection}
\begin{table}[H]
	\centering
	\begin{tabular}{cc}
		\hline
		$ \sigma_h $ & horizontal\\
		$ \sigma_v $ & vertical\\
		$ \sigma_d $ & diagonal (vertical and bisects the angle between $ C_2 $ axis)\\
		\hline
	\end{tabular}
	\caption{}
\end{table}
\paragraph{Inversion}
\paragraph{Rotation Reflection}
\begin{equation}\label{key}
\hat{S}_n = \hat\sigma_h \cross \hat{C}_n
\end{equation}

\subsubsection{Point Groups of Interest to Chemists}
\begin{table}[H]
	\centering
	\begin{tabular}{ccc}
		\hline
		$ C_{nv} $ & &  \\
		$ C_{nh} $ &  & Rotation by $ 360\textdegree/n $\\
		$ D_{nh} $ & &\\
		$ D_{nv} $ &&\\
		$ D_{nd} $ &&\\
		$ T_d $ &&\\
		\hline
	\end{tabular}
	\caption{Symmetry elements and operators}
\end{table}

\subsection{The Symm Operators of a Mol Form a Group}
A set of operators form a group if they satisfy:
\begin{enumerate}
	\item closed under multiplication 乘法封闭
	\item associative multiplication 乘法结合律
	\item only one identity operator 单位元
	\item everyone has only one inverse 逆元
\end{enumerate}

\subsubsection{Point Group for Some Mols}
\paragraph{No Symm Axis}~\\
$ C_1 $ -- nothing\\
$ C_s $ -- $ \sigma $\\
$ C_i $ -- $ i $\\
\paragraph{$ C_n $}
\paragraph{$ S_n $}
\paragraph{$ C_{nv} $} -- $ C_n $ and $ n\sigma_v $
\paragraph{$ C_{nh} $} -- $ C_n $ and $ \sigma_h $
\paragraph{$ D_n $} -- $ C_n $ and $ nC_2 \perp C_n$\\
e.g. 一点点交错的$ \ce{C_3H_6} $, $ C_2 $在3个角平分线处
\paragraph{$ D_{nd} $} -- $ C_n $(also $ S_{2n} $) and $ nC_2 \perp C_n$ and $ n\sigma_d $ 
\paragraph{$ D_{nh} $} -- $ C_n $ and $ nC_2 \perp C_n$ and $ \sigma_h $ 
\paragraph{$ T_d $} 主轴是$ S_4 $
\paragraph{$ O_h $}
\paragraph{$ I_h $}

\subsection{Symm Operators Can Be Represented by Matrices}

\subsection{The $ C_{3v} $ Point Group Has a 2-D Irreducible Representation}

\subsection{The Most Important Summary of the Properties of a Point Group Is Its Character Table}

\paragraph{basis}

\paragraph{class}
same characters -- in a class.\\
\# of class = \# of irred represtn.
\paragraph{notations}
\begin{enumerate}
	\item $ A $:, $ B $:, $ E $:2D, $ T $:3D
	\item $ A_1 $: symm wrt $ C_2/\sigma_v $, $ A_2 $: antisymm wrt that.
	\item $ A' $: symm wrt $ \sigma_h $, $ A'' $: antisymm wrt that.
	\item $ A_g $:, $ A_u $:
\end{enumerate}

\subsection{Several Mathematical Relations Involve the Characters of Irreducible Representation}
\paragraph{notations}~\\
\begin{table}[H]
	\centering
	\begin{tabular}{ccc}
		\hline
		XU G.X. & McQuarrie &  \\ \hline
		$ D^{(\nu)}(R) $ &  & \\
		$ \chi^{(\nu)}(R) $ & $ \chi_j(R) $ &\\
		$ n_\nu $ & $ d_j $ & dimension of repr matrix \\
		$ a_\nu $ & $ a_j $ &\\
		$ g $ & $ h $ &\\
		\hline
	\end{tabular}
	\caption{}
\end{table}
\paragraph{order}
\begin{equation}\label{key}
\sum_\nu n_\nu^2 = g
\end{equation}
\paragraph{character}
\begin{equation}\label{key}
\sum_R D_{il}^{(\nu)} D_{jm}^{*(\mu)} = \dfrac{g}{n_\nu} \delta_{\mu\nu}\delta_{ij}\delta_{lm}
\end{equation}
\begin{equation}\label{key}
\sum_{R} \chi^{(\nu)}(R)\chi^{*(\mu)}(R) = g\delta_{\mu\nu}
\end{equation}
\begin{equation}\label{key}
\sum_R \chi^{(\nu)}(R) = 0 \quad (\nu \neq A_1)
\end{equation}
\paragraph{reduce a given reducible repr $ \Gamma $}~\\
Suppose
\begin{equation}\label{key}
\chi(R) = \sum_\nu a_\nu \chi^{(\nu)}(R)
\end{equation}
thus
\begin{equation}\label{key}
a_\nu = \dfrac{1}{g}\sum_R\chi(R)\chi^{(\nu)}(R)
\end{equation}

\subsection{Use Symm Arguments to Predict Which Elements in a Secular Det Equals 0}

\subsection{Generating Operators Are Used to Find LCAOs That Are Bases for IrRepr}
\begin{equation}\label{key}
\hP_j = \dfrac{d_j}{h}\sum_{\hR} \chi_j(\hR)\hR
\end{equation}

\section{Molecular Spectroscopy}
\subsection{}
\begin{table}[H]
	\centering
	\begin{tabular}{ccccc}
		\hline
		& micro & far IR & IR & visible \& UV  \\ \hline
		$ f/\si{Hz} $ &  & & &\\
		$ \lambda/\si{m} $ &  & & &\\
		$ \bar{\nu}/\si{cm^{-1}} $ &&&&\\
		$ E/\si{J.mol^{-1}} $ &  & & &\\
		process &  & & &\\
		\hline
	\end{tabular}
	\caption{}
\end{table}

\subsection{Rotational Transitions Accompany Vibrational Transitions}
Vib \& rot energy
\begin{align}\label{key}
\tilde{E} &= G(\nu) + F(J) \\
&= (v + 1/2)\tilde{\nu} + \tilde{B}J(J+1)
\end{align}
selex rule:
\begin{equation}\label{key}
\Delta v = \pm 1 \quad \Delta J = \pm 1
\end{equation}
P -- left -- $ \Delta J = -1 $ -- wide\\
R -- right -- $ \Delta J = +1 $ -- narrow

\subsection{}
\begin{equation}\label{key}
\tilde{B}_v = \tilde{B}_e - \tilde{\alpha}_e(v + 1/2)
\end{equation}
\begin{equation}\label{key}
\tilde{B}_0 > \tilde{B}_1 > \cdots
\end{equation}
which makes P-branches wider.

\subsection{}
\begin{equation}\label{key}
F(J) = \tilde{B}J(J+1) - \tilde{D}J^2 (J+1)^2
\end{equation}
$ \tilde{D} $: 

\subsection{Overtones Are Observed in Vibrational Spectra}
\begin{equation}\label{key}
G(v) = \tilde\nu_e \qty(v + \dfrac{1}{2}) - \tilde{x}_e\tilde{\nu}_e\qty(v + \dfrac{1}{2})^2
\end{equation}
$ \tilde{x}_e $: anharmonicity cons.\\
\begin{equation}\label{key}
 \tilde{v}_{obs} = \tilde{\nu}_e v - \tilde{x}_e\tilde{\nu}_e v(v+1) 
\end{equation}
$ v = 1,2,... $

\subsection{Electronic Spectra Contains Electronic, Vibrational and Rotational Info}
\begin{equation}\label{key}
\tilde{E} = \tilde{nu}_{el} + \tilde{\nu}_e(v +1/2) - \tilde{x}_e\tilde{\nu}_e(v + 1/2)^2 + \cdots(rot)
\end{equation}
vibronic transitions:\\
$ 0\ra 0 $
\begin{equation}\label{key}
\tilde{v}_{0,0} = \tilde{T}_e + \dfrac{1}{2}(\tilde{\nu}_e' - \tilde{\nu}_e'') - \dfrac{1}{4}(\tilde{x}_e'\tilde{\nu}' - \tilde{x}_e''\tilde{\nu}'')
\end{equation}

\subsection{Franck-Condon Principle Predicts the Relative Intensities of Vibronic Transitions}

\subsection{The Rotational Spectrum of a Polyatomic Mols Depends Upon the Principal Moments of Inertia of the Mol}
\begin{equation}\label{key}
\mqty(I_{xx} & I_{xy} & I_{xz}\\ I_{xy} & I_{yy} & I_{yz} \\ I_{xz} & I_{yz} & I_{zz}) \xrightarrow{\text{diagnalization}} \mqty(I_A && \\ & I_B &\\ && I_C)
\end{equation}
\begin{table}[H]
	\centering
	\begin{tabular}{ccc}
		\hline
		& top & requisition \\ \hline
		$ I_C = I_B > I_A = 0 $ &  & \\
		$ I_C = I_B = I_A $ & sph top & $ 2C_n, n\geq 3 $\\
		$ I_C = I_B > I_A $ & prolate symm top &\\
		$ I_C > I_B = I_A $ & oblate symm top & \\
		$ I_C \neq I_B \neq I_A $& asymm & \\
		\hline
	\end{tabular}
	\caption{}
\end{table}

\subsection{The Vibrations of Polyatomic Mols Are Represented by Normal Coordinates}

\subsection{Normal Coordinates Belong to Irreducible Representations of Mol Point Groups}
Contribution to $ \chi(R) $ per unmoved atom
\begin{table}[H]
	\centering
	\begin{tabular}{cc}
		\hline
		$ \hR $ & contribution per unmoved atom \\ \hline
		& \\
		\hline
	\end{tabular}
	\caption{}
\end{table}

Now we get $ \Gamma_{3N} $.\\
Subtract the irreducible representations corresponding to translational ($ x,y,z $) and rotational ($ R_x, R_y, R_z $) degrees of freedom, we get $ \Gamma_{vib} $.

\subsection{Selection Rules Are Derived from TD Perturbation Theory}
Consider a mol interacting w/ EM radiation. The EM field
\begin{equation}\label{key}
\vb{E} = \vb{E}_0\cos 2\pi\nu t
\end{equation}
\begin{equation}\label{key}
\hH^{(1)} = -\bm\mu\cdot\vb{E} = -\bm\mu\vb{E}_0\qty(\e^{\I 2\pi\nu t} + \e^{-\I 2\pi\nu t})/2
\end{equation}
\begin{equation}\label{key}
\Psi(t) = a_1(t)\Psi_1(t) + a_2(t)\Psi_2(t)
\end{equation}
\begin{equation}\label{key}
a_1(t)\hH^{(1)}\Psi_1 + a_2(t)\hH^{(1)}\Psi_2 = \I\hbar\qty(\Psi_1\dv{a_1}{t} + \Psi_2\dv{a_2}{t})
\end{equation}
\begin{equation}\label{key}
a_1(t)\Braket{\psi_2 | \hH^{(1)} | \Psi_1} + a_2(t)\Braket{\psi_2 | \hH^{(1)} | \Psi_2} = \I\hbar\qty(0 + \dv{a_2}{t}\e^{-\I Et/\hbar})
\end{equation}
...
\begin{equation}\label{key}
\I\hbar\dv{a_2}{t} = \e^{-\I(E_1 - E_2)t/\hbar} \Braket{\psi_2 | \hH^{(1)} | \psi_1}
\end{equation}
\begin{equation}\label{key}
\dv{a_2}{t} \approx ...
\end{equation}

\subsection{The Selection Rule in the Rigid-Rotator Approx Is $ \Delta J = \pm 1 $}
\begin{align}
\Braket{J',M' | \mu_z | J,M} &= \int_0^{2\pi} \dd\phi \int_0^\pi Y_{J'}^{M'*} \mu_z Y_J^M \sin\theta\dd\theta \notag\\
&= ...
\end{align}

\subsection{The Harmonic-Oscillator Selection Rule Is $ \Delta\nu = \pm 1 $}

\section{Nuclear Magnetic Resonance Spectroscopy}
\subsection{Nuclei Have Intrinsic Spin Angular Momenta}
\subsection{Magnetic Moments Interact with Magnetic Fields}
\subsection{}
\subsection{The Magnetic Field Acting upon Nuclei in Mols Is Shielded}
\subsection{Chemical Shifts Depend upon the Chemical Environment of the Nucleus}
\subsection{Spin-Spin Coupling}
\begin{equation}\label{key}
\hH = -\gamma B_0(1-\sigma_1)\hat{\vb{I}}_{z1} -\gamma B_0(1-\sigma_2)\hat{\vb{I}}_{z2} + {2\pi J_{12}}\hat{\vb{I}}_1 \cdot \hat{\vb{I}}_2
\end{equation}
\begin{equation}\label{key}
\hH^{(0)} = -\gamma B_0(1-\sigma_1)\hat{\vb{I}}_{z1} -\gamma B_0(1-\sigma_2)\hat{\vb{I}}_{z2} \quad \hH^{(1)} {2\pi J_{12}}\hat{\vb{I}}_1 \cdot \hat{\vb{I}}_2
\end{equation}
\begin{align}
\psi_1^{(0)} &= \alpha(1)\alpha(2) & \psi_2^{(0)} &= \beta(1)\alpha(2)\\
\psi_3^{(0)} &= \alpha(1)\beta(2) & \psi_4^{(0)} &= \beta(1)\beta(2)
\end{align}
\begin{align}
E_1^{(0)} &= -\gamma B_0\qty(1 - \dfrac{\sigma_1 + \sigma_2}{2}) &
E_2^{(0)} &= -\gamma B_0(\sigma_1 - \sigma_2)\\
E_3^{(0)} &= \gamma B_0(\sigma_1 - \sigma_2) &
E_4^{(0)} &= \gamma B_0\qty(1 - \dfrac{\sigma_1 + \sigma_2}{2})
\end{align}
perturbed to 1st order
\begin{align}
E_1 &= -\gamma B_0\qty(1 - \dfrac{\sigma_1 + \sigma_2}{2}) + \dfrac{2\pi J_{12}}{4} \\
E_2 &= -\gamma B_0(\sigma_1 - \sigma_2) - \dfrac{2\pi J_{12}}{4}\\
E_3 &= \gamma B_0(\sigma_1 - \sigma_2) - \dfrac{2\pi J_{12}}{4} \\
E_4 &= \gamma B_0\qty(1 - \dfrac{\sigma_1 + \sigma_2}{2}) + \dfrac{2\pi J_{12}}{4}
\end{align}
Since
\begin{equation}\label{key}
\nu_0 = \dfrac{\gamma B_0}{2\pi}
\end{equation}
\begin{align}
\nu_{1\ra 2} &= \nu_0(1 - \sigma_1) - \dfrac{J_{12}}{2}\\
\nu_{1\ra 3} &= \nu_0(1 - \sigma_2) - \dfrac{J_{12}}{2}\\
\nu_{2\ra 4} &= \nu_0(1 - \sigma_2) + \dfrac{J_{12}}{2}\\
\nu_{3\ra 4} &= \nu_0(1 - \sigma_1) + \dfrac{J_{12}}{2}
\end{align}

\subsection{Spin-Spin Coupling Between Chemically Equivalent Protons}

\subsection{The $ n+1 $ Rule}

\subsection{2nd-Order Spectra}
Only for the case in which 
\begin{equation}\label{key}
J << \nu_0\abs{\sigma_1 - \sigma_2}
\end{equation}
the $ n+1 $ spectra.

\section{Lasers, Laser Spectroscopy and Photochemistry}
\subsection{}
\subsection{The Dynamics of Spectroscopic Transitions between the Electronic States}
\paragraph{absorption}
\begin{equation}\label{key}
-\dv{N_1(t)}{t} = B_{12}\rho_\nu(\nu_{12})N_1(t)
\end{equation}
\paragraph{emission}

\subsection{Population Inversion, 2-Level System}
\subsection{Population Inversion, 3-Level System}
\subsection{What is Inside a Laser?}
\subsection{He-Ne Laser}
\subsection{High-Resolution Laser Spectroscopy}
Hyperfine structure
\subsection{The Dynamics of Photochemistry Process}
def: quantum yield
\begin{equation}\label{key}
\Phi = \dfrac{\text{\# mols undergoing reaction}}{\text{\# photons absorbed}}
\end{equation}

\section{The Properties of Gases}

\section{The Boltzmann Factor and Partition Functions}
\subsection{}
\subsection{Partition Function}
\begin{equation}\label{key}
Q = 
\end{equation}
\begin{equation}\label{key}
p_j = \dfrac{\e^{-E_j\beta}}{Q}
\end{equation}

\subsection{Thermodynamic Quantities}
\subsubsection{Energy}

\subsubsection{Work and Heat}

\subsection{Pressure}

\subsubsection{Heat Capacity}

\subsubsection{Entropy}

\subsection{}

\subsection{The Partition Function of a System of Independent, Distinguishable Mols}
\begin{equation}\label{key}
Q(N,V,T) = \sum_{i,j,k,...} \e^{-\beta(\varepsilon_i +\varepsilon_j + \varepsilon_k)} = \sum_i\e^{-\beta\varepsilon_i} \sum_j\e^{-\beta\varepsilon_j}\cdots
\end{equation}
\subsection{The Partition Function of a System of Independent, Indistinguishable Mols}
\begin{equation}\label{key}
Q(N,V,T) = \dfrac{q(V,T)^N}{N!}
\end{equation}


\subsection{}
\begin{equation}\label{key}
\ev{\varepsilon} = \sum_j \dfrac{\varepsilon_j \e^{-\beta\varepsilon_j}}{q}
\end{equation}
\begin{equation}\label{key}
\varepsilon = \varepsilon_i^{trans} + \varepsilon_j^{rot} + \varepsilon_k^{vib} + \varepsilon_l^{elec}
\end{equation}
\begin{equation}\label{key}
\pi_{ijkl} = \dfrac{\e^{-\beta\varepsilon_i^{trans}}\e^{-\beta\varepsilon_j^{rot}} \e^{-\beta\varepsilon_k^{vib}}\e^{-\beta\varepsilon_l^{elec}}}{q_{trans}q_{rot}q_{vib}q_{elec}}
\end{equation}

\section{Partition Functions and Ideal Gases}
\subsection{The Translational PF of Monatomic Ideal Gas}
\begin{equation}\label{key}
q_{\trans} = \qty(\dfrac{2\pi m \kB T}{h^2})^{3/2} T
\end{equation}

\subsection{The Electronic PF}

\subsection{}

\subsection{The Rotational PF}
\begin{equation}\label{key}
q_{\rot} = \sum_{J=0} (2J+1)\e^{-J(J+1)\Theta_{\rot}/T}
\end{equation}
where
\begin{equation}\label{key}
\Theta_{rot} = \dfrac{\hbar^2}{2I\kB}
\end{equation}
At low $ \Theta_{\rot} $
\begin{equation}\label{key}
q_{\rot} = ... = \dfrac{T}{\Theta_{\rot}}
\end{equation}

\paragraph{Symmetry Number}
\begin{equation}\label{key}
q_{\rot} = \dfrac{T}{\sigma \Theta_{\rot}}
\end{equation}
Spherical top
\begin{equation}\label{key}
q_{\rot} = \dfrac{\sqrt{\pi}}{\sigma}\qty(\dfrac{T}{\Theta_{\rot}})^{3/2}
\end{equation}

\setcounter{section}{24}
\section{The Kinetic Theory of Gases}
\subsection{}
\subsection{Speed Distribution}
\begin{equation}\label{key}
f(u_x) = \sqrt{\dfrac{m}{2\pi\kB T}}\e^{-m u_x^2/2\kB T}
\end{equation}
\begin{equation}\label{key}
\ev{u_x^2} = \dfrac{\kB T}{m} = \dfrac{R T}{M}
\end{equation}

\subsection{Maxwell Distribution}
\begin{equation}\label{key}
F(u) = 4\pi u^2 \qty(\dfrac{m}{2\pi\kB T})^{3/2}\e^{-m u^2/2\kB T}
\end{equation}
\begin{equation}\label{key}
\ev{u} = \sqrt{\dfrac{8\kB T}{\pi m}}
\end{equation}
\begin{equation}\label{key}
\ev{u^2} = \dfrac{3\kB T}{m}
\end{equation}
\begin{equation}\label{key}
u_{mp} = \sqrt{\dfrac{2\kB T}{m}}
\end{equation}
\begin{equation}\label{key}
F(\varepsilon) = \dfrac{2\pi}{(\pi \kB T)^{3/2}}\varepsilon^{1/2}\e^{-\varepsilon/\kB T}
\end{equation}
\begin{equation}\label{key}
\ev{\varepsilon} = \dfrac{3}{2}\kB T
\end{equation}

\subsection{The Frequency of Collisions with a Wall}
\begin{equation}\label{key}
\dd z = \dfrac{1}{A}\dv{N}{t}
\end{equation}
freq per area
\begin{equation}\label{key}
z = \dfrac{\rho}{4}\ev{u}
\end{equation}

\subsection{Inter-collision and MFP}
\begin{equation}\label{key}
z_A  = \rho \sigma \ev{u_r} = \rho\sigma \sqrt{2}\ev{u}
\end{equation}
\begin{equation}\label{key}
l = \dfrac{u}{z_A} = \dfrac{1}{\sqrt{2}\rho\sigma}
\end{equation}
\begin{equation}\label{key}
p(x)\dd x = \dfrac{1}{l}\e^{-x/l}\dd x
\end{equation}


\end{document}