%\documentclass[UTF8]{ctexart} % use larger type; default would be 10pt
\documentclass[a4paper]{article}
\usepackage{xeCJK}
%\usepackage{ctex}
%\usepackage{luatexja-fontspec}
%\setmainjfont{FandolSong}
%\usepackage[utf8]{inputenc} % set input encoding (not needed with XeLaTeX)

%%% Examples of Article customizations
% These packages are optional, depending whether you want the features they provide.
% See the LaTeX Companion or other references for full information.

%%% PAGE DIMENSIONS
\usepackage{geometry} % to change the page dimensions
\geometry{a4paper} % or letterpaper (US) or a5paper or....
% \geometry{margin=2in} % for example, change the margins to 2 inches all round
% \geometry{landscape} % set up the page for landscape
%   read geometry.pdf for detailed page layout information

\usepackage{graphicx} % support the \includegraphics command and options

% \usepackage[parfill]{parskip} % Activate to begin paragraphs with an empty line rather than an indent

%%% PACKAGES
\usepackage{booktabs} % for much better looking tables
\usepackage{array} % for better arrays (eg matrices) in maths
\usepackage{paralist} % very flexible & customisable lists (eg. enumerate/itemize, etc.)
\usepackage{verbatim} % adds environment for commenting out blocks of text & for better verbatim
\usepackage{subfig} % make it possible to include more than one captioned figure/table in a single float
% These packages are all incorporated in the memoir class to one degree or another...

%%% HEADERS & FOOTERS
\usepackage{fancyhdr} % This should be set AFTER setting up the page geometry
\pagestyle{fancy} % options: empty , plain , fancy
\renewcommand{\headrulewidth}{0pt} % customise the layout...
\lhead{}\chead{}\rhead{}
\lfoot{}\cfoot{\thepage}\rfoot{}

%%% SECTION TITLE APPEARANCE
\usepackage{sectsty}
\allsectionsfont{\sffamily\mdseries\upshape} % (See the fntguide.pdf for font help)
% (This matches ConTeXt defaults)

%%% ToC (table of contents) APPEARANCE
\usepackage[nottoc,notlof,notlot]{tocbibind} % Put the bibliography in the ToC
\usepackage[titles,subfigure]{tocloft} % Alter the style of the Table of Contents
\renewcommand{\cftsecfont}{\rmfamily\mdseries\upshape}
\renewcommand{\cftsecpagefont}{\rmfamily\mdseries\upshape} % No bold!

%%% END Article customizations

%%% The "real" document content comes below...

\setlength{\parindent}{0pt}
\usepackage{physics}
\usepackage{amsmath}
%\usepackage{symbols}
\usepackage{AMSFonts}
\usepackage{bm}
%\usepackage{eucal}
\usepackage{mathrsfs}
\usepackage{amssymb}
\usepackage{float}
\usepackage{multicol}
\usepackage{abstract}
\usepackage{empheq}
\usepackage{extarrows}
\usepackage{textcomp}
\usepackage{mhchem}
\usepackage{braket}
\usepackage{siunitx}
\usepackage[utf8]{inputenc}
\usepackage{tikz-feynman}
\usepackage{feynmp}


\DeclareMathOperator{\p}{\prime}
\DeclareMathOperator{\ti}{\times}

\DeclareMathOperator{\e}{\mathrm{e}}
\DeclareMathOperator{\I}{\mathrm{i}}
\DeclareMathOperator{\Arg}{\mathrm{Arg}}
\newcommand{\NA}{N_\mathrm{A}}
\newcommand{\kB}{k_\mathrm{B}}

\DeclareMathOperator{\ra}{\rightarrow}
\DeclareMathOperator{\llra}{\longleftrightarrow}
\DeclareMathOperator{\lra}{\longrightarrow}
\DeclareMathOperator{\dlra}{\;\Leftrightarrow\;}
\DeclareMathOperator{\dra}{\;\Rightarrow\;}

%%%%%%%%%%%% QUANTUM MECHANICS %%%%%%%%%%%%%%%%%%%%%%%%
\newcommand{\bkk}[1]{\Braket{#1|#1}}
\newcommand{\bk}[2]{\Braket{#1|#2}}
\newcommand{\bkev}[2]{\Braket{#2|#1|#2}}

\DeclareMathOperator{\na}{\bm{\nabla}}
\DeclareMathOperator{\nna}{\nabla^2}
\DeclareMathOperator{\drrr}{\dd[3]\vb{r}}

\DeclareMathOperator{\psis}{\psi^\ast}
\DeclareMathOperator{\Psis}{\Psi^\ast}
\DeclareMathOperator{\hi}{\hat{\vb{i}}}
\DeclareMathOperator{\hj}{\hat{\vb{j}}}
\DeclareMathOperator{\hk}{\hat{\vb{k}}}
\DeclareMathOperator{\hr}{\hat{\vb{r}}}
\DeclareMathOperator{\hT}{\hat{\vb{T}}}
\DeclareMathOperator{\hH}{\hat{H}}

\DeclareMathOperator{\hL}{\hat{\vb{L}}}
\DeclareMathOperator{\hp}{\hat{\vb{p}}}
\DeclareMathOperator{\hx}{\hat{\vb{x}}}
\DeclareMathOperator{\ha}{\hat{\vb{a}}}
\DeclareMathOperator{\hS}{\hat{\vb{S}}}
\DeclareMathOperator{\hSigma}{\hat{\bm\Sigma}}
\DeclareMathOperator{\hJ}{\hat{\vb{J}}}

\DeclareMathOperator{\Tdv}{-\dfrac{\hbar^2}{2m}\dv[2]{x}}
\DeclareMathOperator{\Tna}{-\dfrac{\hbar^2}{2m}\nabla^2}

%\DeclareMathOperator{\s}{\sum_{n=1}^{\infty}}
\DeclareMathOperator{\intinf}{\int_0^\infty}
\DeclareMathOperator{\intdinf}{\int_{-\infty}^\infty}
%\DeclareMathOperator{\suminf}{\sum_{n=0}^\infty}
\DeclareMathOperator{\sumnzinf}{\sum_{n=0}^\infty}
\DeclareMathOperator{\sumnoinf}{\sum_{n=1}^\infty}
\DeclareMathOperator{\sumndinf}{\sum_{n=-\infty}^\infty}
\DeclareMathOperator{\sumizinf}{\sum_{i=0}^\infty}

%%%%%%%%%%%%%%%%% PARTICLE PHYSICS %%%%%%%%%%%%%%%%
\DeclareMathOperator{\hh}{\hat{h}}               % helicity
\DeclareMathOperator{\hP}{\hat{\vb{P}}}          % Parity
\DeclareMathOperator{\hU}{\hat{U}}
\DeclareMathOperator{\hG}{\hat{G}}

\DeclareMathOperator{\GeV}{\si{GeV}}
\DeclareMathOperator{\LI}{\mathscr{L}.I.}
%\DeclareMathOperator{\g5}{\gamma^5}
\DeclareMathOperator{\gmuu}{\gamma^\mu}
\DeclareMathOperator{\gmud}{\gamma_\mu}
\DeclareMathOperator{\gnuu}{\gamma^\nu}
\DeclareMathOperator{\gnud}{\gamma_\nu}

\renewcommand{\u}{\mathrm{u}}
\renewcommand{\d}{\mathrm{d}}
\DeclareMathOperator{\s}{\mathrm{s}}

\DeclareMathOperator{\q}{\mathrm{q}}
\DeclareMathOperator{\bq}{\bar{\mathrm{q}}}

\DeclareMathOperator{\g}{\mathrm{g}}
\DeclareMathOperator{\W}{\mathrm{W}}
\DeclareMathOperator{\Z}{\mathrm{Z}}

%%% Feynman Diagram
\newcommand{\pa}{particle}
\newcommand{\mo}{momentum}
\newcommand{\el}{edge label}

%%% APC
\DeclareMathOperator{\hR}{\hat{\vb{R}}}
\DeclareMathOperator{\trans}{\mathrm{trans}}
\DeclareMathOperator{\rot}{\mathrm{rot}}
\DeclareMathOperator{\vib}{\mathrm{vib}}
\DeclareMathOperator{\elec}{\mathrm{elec}}

\newcommand{\dis}{\displaystyle}
\numberwithin{equation}{section}
\setcounter{tocdepth}{4}

\title{Notes of \textbf{Advanced Physical Chemistry II}}
\author{hebrewsnabla}
%\date{} % Activate to display a given date or no date (if empty),
         % otherwise the current date is printed 

\begin{document}
% \boldmath
\maketitle

\tableofcontents

\newpage


\setcounter{section}{24}
\section{The Kinetic Theory of Gases}
\subsection{}
\subsection{Speed Distribution}
\begin{equation}\label{key}
f(u_x) = \sqrt{\dfrac{m}{2\pi\kB T}}\e^{-m u_x^2/2\kB T}
\end{equation}
\begin{equation}\label{key}
\ev{u_x^2} = \dfrac{\kB T}{m} = \dfrac{R T}{M}
\end{equation}

\subsection{Maxwell Distribution}
\begin{equation}\label{key}
F(u) = 4\pi u^2 \qty(\dfrac{m}{2\pi\kB T})^{3/2}\e^{-m u^2/2\kB T}
\end{equation}
\begin{equation}\label{key}
\ev{u} = \sqrt{\dfrac{8\kB T}{\pi m}}
\end{equation}
\begin{equation}\label{key}
\ev{u^2} = \dfrac{3\kB T}{m}
\end{equation}
\begin{equation}\label{key}
u_{mp} = \sqrt{\dfrac{2\kB T}{m}}
\end{equation}
\begin{equation}\label{key}
F(\varepsilon) = \dfrac{2\pi}{(\pi \kB T)^{3/2}}\varepsilon^{1/2}\e^{-\varepsilon/\kB T}
\end{equation}
\begin{equation}\label{key}
\ev{\varepsilon} = \dfrac{3}{2}\kB T
\end{equation}

\subsection{The Frequency of Collisions with a Wall}
\begin{equation}\label{key}
\dd z = \dfrac{1}{A}\dv{N}{t}
\end{equation}
freq per area
\begin{equation}\label{key}
z = \dfrac{\rho}{4}\ev{u}
\end{equation}

\subsection{}
\subsection{Inter-collision and MFP}
\begin{equation}\label{key}
z_A  = \rho \sigma \ev{u_r} = \rho\sigma \sqrt{2}\ev{u}
\end{equation}
\begin{equation}\label{key}
l = \dfrac{u}{z_A} = \dfrac{1}{\sqrt{2}\rho\sigma}
\end{equation}
\begin{equation}\label{key}
p(x)\dd x = \dfrac{1}{l}\e^{-x/l}\dd x
\end{equation}








\section{Chemical Kinetics I: Rate Laws}
\setcounter{subsection}{7}
\subsection{Reaction Rate Constants}
\begin{equation}\label{key}
k(T) = \dfrac{\kB T}{h c^\circ}\e^{-\Delta^\ddagger G^\circ/RT}
\end{equation}

\section{}


\section{Gas Phase Reaction Dynamics}
\subsection{Hard-sphere Collision Theory}
Na\"ive hard-sphere rate cons.
\begin{equation}\label{key}
k = \sigma_{AB} \NA \ev{u_r}     \quad (\si{dm^3.mol^{-1}.s^{-1}})
\end{equation}
experimental
\begin{equation}\label{key}
k = A\e^{-E_a/RT}
\end{equation}
Taking into account the dependence of the rate on $ \ev{u_r} $\\
\begin{equation}\label{key}
\sigma_r(E_r) = \left\{
\mqty{0            & E_r<E_0\\
	  \pi d_{AB}^2 & E_r \geq E_0}\right.
\end{equation}
...
\begin{equation}\label{key}
k = \sigma_{AB}\NA \ev{u_r} \e^{-E_0/\kB T}\qty(1 + \dfrac{E_0}{\kB T})
\end{equation}

\subsection{Reaction Cross Section Depending on the Impact Parameter}
\paragraph{line-of-centers model}~\\
energy dependence of the reaction CS
\begin{equation}\label{key}
\sigma_r(E_r) = \left\{
\mqty{0            & E_r<E_0\\
	\pi d_{AB}^2\qty(1-\dfrac{E_0}{E_r}) & E_r \geq E_0}\right.
\end{equation}
\begin{equation}\label{LoC_k}
k = \sigma_{AB}\NA \ev{u_r} \e^{-E_0/\kB T}
\end{equation}
\paragraph{Example 28-3}
\begin{equation}\label{key}
k = A\e^{-E_a/\kB T} \dra E_a = \kB T^2\dv{\ln k}{T}
\end{equation}
...
\begin{equation}\label{key}
E_a = E_0 + \dfrac{1}{2}\kB T
\end{equation}
with $ \eqref{LoC_k} $,
\begin{equation}\label{key}
\sigma_{AB}\NA \ev{u_r} = A\e^{-1/2} \dra A = \sigma_{AB}\NA \ev{u_r}\e^{1/2}
\end{equation}

\subsection{}

\subsection{The Internal Energy of the Reactants}

\subsection{CoM Coordinate System}
\begin{equation}\label{key}
\text{KE} = \dfrac{1}{2}m_Au_A^2 +\dfrac{1}{2}m_Bu_B^2 = \dfrac{1}{2}M u_{CoM}^2 + \dfrac{1}{2}\mu u_r^2
\end{equation}
Since $ u_{CoM} $ is conserved,
\begin{equation}\label{key}
U(R) + \dfrac{1}{2}\mu u_r^2(R) = U(P) + \dfrac{1}{2}\mu u_r^2(P)
\end{equation}

\subsection{}

\subsection{Vibrationally Excited Products}
\begin{equation}\label{key}
E_{tot} = E_{trans} + E_{vib} + E_{elec}
\end{equation}
\begin{equation}\label{key}
E_{vib} = \tilde{\nu}_e(v+1/2) - \tilde{\nu}_e\tilde{x}_e(v+1/2)^2
\end{equation}
\begin{equation}\label{key}
E_{rot} = [\tilde{B}_e - \tilde\alpha_e(v+1/2)]J(J+1)
\end{equation}

\subsection{Velocity and Angular Distribution}

\subsection{}
\subsection{}

\section{Solids and Surface Chemistry}
\subsection{}

\subsection{Miller Indices}
\begin{equation}\label{key}
\dfrac{1}{d^2} = \dfrac{h^2 + k^2 + l^2}{a^2}
\end{equation}

\subsection{XRD}


\subsection{Intensity}


\end{document}