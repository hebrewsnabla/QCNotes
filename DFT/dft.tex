%\documentclass[UTF8]{ctexart} % use larger type; default would be 10pt
\documentclass[a4paper]{article}
\usepackage{xeCJK}
%\usepackage[utf8]{inputenc} % set input encoding (not needed with XeLaTeX)

%%% Examples of Article customizations
% These packages are optional, depending whether you want the features they provide.
% See the LaTeX Companion or other references for full information.

%%% PAGE DIMENSIONS
\usepackage{geometry} % to change the page dimensions
\geometry{a4paper} % or letterpaper (US) or a5paper or....
\geometry{margin=1in} % for example, change the margins to 2 inches all round
% \geometry{landscape} % set up the page for landscape
%   read geometry.pdf for detailed page layout information

\usepackage{graphicx} % support the \includegraphics command and options

% \usepackage[parfill]{parskip} % Activate to begin paragraphs with an empty line rather than an indent

%%% PACKAGES
\usepackage{booktabs} % for much better looking tables
\usepackage{array} % for better arrays (eg matrices) in maths
\usepackage{paralist} % very flexible & customisable lists (eg. enumerate/itemize, etc.)
\usepackage{verbatim} % adds environment for commenting out blocks of text & for better verbatim
\usepackage{subfig} % make it possible to include more than one captioned figure/table in a single float
% These packages are all incorporated in the memoir class to one degree or another...

%%% HEADERS & FOOTERS
\usepackage{fancyhdr} % This should be set AFTER setting up the page geometry
\pagestyle{fancy} % options: empty , plain , fancy
\renewcommand{\headrulewidth}{0pt} % customise the layout...
\lhead{}\chead{}\rhead{}
\lfoot{}\cfoot{\thepage}\rfoot{}

%%% SECTION TITLE APPEARANCE
\usepackage{sectsty}
\allsectionsfont{\sffamily\mdseries\upshape} % (See the fntguide.pdf for font help)
% (This matches ConTeXt defaults)

%%% ToC (table of contents) APPEARANCE
\usepackage[nottoc,notlof,notlot]{tocbibind} % Put the bibliography in the ToC
\usepackage[titles,subfigure]{tocloft} % Alter the style of the Table of Contents
\renewcommand{\cftsecfont}{\rmfamily\mdseries\upshape}
\renewcommand{\cftsecpagefont}{\rmfamily\mdseries\upshape} % No bold!

%%% END Article customizations

%%% The "real" document content comes below...

\setlength{\parindent}{0pt}
\usepackage{physics}
\usepackage{amsmath}
%\usepackage{symbols}
\usepackage{AMSFonts}
\usepackage{bm}
%\usepackage{eucal}
\usepackage{mathrsfs}
\usepackage{amssymb}
\usepackage{float}
\usepackage{multicol}
\usepackage{abstract}
\usepackage{empheq}
\usepackage{extarrows}
\usepackage{textcomp}
\usepackage{fontspec}
\usepackage{braket}
\usepackage{siunitx}
\usepackage{xcolor}
\usepackage{hyperref}
\usepackage{listings}

\setmonofont[Mapping={}]{Consolas}	%英文引号之类的正常显示,相当于设置英文字体
%\setsansfont{Monaco} %设置英文字体 Monaco, Consolas,  Fantasque Sans Mono
%\setmainfont{} %设置英文字体
\definecolor{mygreen}{rgb}{0,0.6,0}
\definecolor{mygray}{rgb}{0.5,0.5,0.5}
\definecolor{mymauve}{rgb}{0.58,0,0.82}
\lstset{ 
	backgroundcolor=\color{white},   % choose the background color; you must add \usepackage{color} or \usepackage{xcolor}; should come as last argument
	basicstyle=\footnotesize\ttfamily,        % the size of the fonts that are used for the code
	breakatwhitespace=false,         % sets if automatic breaks should only happen at whitespace
	breaklines=true,                 % sets automatic line breaking
	captionpos=b,                    % sets the caption-position to bottom
	commentstyle=\color{mygreen},    % comment style
	deletekeywords={...},            % if you want to delete keywords from the given language
	escapeinside={\%*}{*)},          % if you want to add LaTeX within your code
	extendedchars=true,              % lets you use non-ASCII characters; for 8-bits encodings only, does not work with UTF-8
	firstnumber=1,                % start line enumeration with line 1000
	frame=single,	                   % adds a frame around the code
	keepspaces=true,                 % keeps spaces in text, useful for keeping indentation of code (possibly needs columns=flexible)
	keywordstyle=\color{blue},       % keyword style
	language=Octave,                 % the language of the code
	morekeywords={*,...},            % if you want to add more keywords to the set
	numbers=left,                    % where to put the line-numbers; possible values are (none, left, right)
	numbersep=5pt,                   % how far the line-numbers are from the code
	numberstyle=\tiny\color{mygray}, % the style that is used for the line-numbers
	rulecolor=\color{black},         % if not set, the frame-color may be changed on line-breaks within not-black text (e.g. comments (green here))
	showspaces=false,                % show spaces everywhere adding particular underscores; it overrides 'showstringspaces'
	showstringspaces=false,          % underline spaces within strings only
	showtabs=false,                  % show tabs within strings adding particular underscores
	stepnumber=2,                    % the step between two line-numbers. If it's 1, each line will be numbered
	stringstyle=\color{mymauve},     % string literal style
	tabsize=4,	                   % sets default tabsize to 2 spaces
	title=\lstname                   % show the filename of files included with \lstinputlisting; also try caption instead of title
}

\sisetup{
	separate-uncertainty = true,
	inter-unit-product = \ensuremath{{}\cdot{}}
}

\DeclareMathOperator{\p}{\prime}
\DeclareMathOperator{\ti}{\times}
\DeclareMathOperator{\intinf}{\int_0^\infty}
\DeclareMathOperator{\intdinf}{\int_{-\infty}^\infty}
\DeclareMathOperator{\intzpi}{\int_0^\pi}
\DeclareMathOperator{\intztpi}{\int_0^{2\pi}}
\DeclareMathOperator{\suminf}{\sum_{n=1}^{\infty}}
\DeclareMathOperator{\suminfz}{\sum_{n=0}^\infty}
\DeclareMathOperator{\sumkinf}{\sum_{k=1}^{\infty}}
\DeclareMathOperator{\sumkinfz}{\sum_{k=0}^\infty}
\DeclareMathOperator{\e}{\mathrm{e}}
\DeclareMathOperator{\I}{\mathrm{i}}
\DeclareMathOperator{\Arg}{\mathrm{Arg}}
\DeclareMathOperator{\ra}{\rightarrow}
\DeclareMathOperator{\llra}{\longleftrightarrow}
\DeclareMathOperator{\lra}{\longrightarrow}
\DeclareMathOperator{\dlra}{\Leftrightarrow}
\DeclareMathOperator{\dra}{\Rightarrow}
\newcommand{\bkk}[1]{\Braket{#1|#1}}
\newcommand{\bk}[2]{\Braket{#1|#2}}
\newcommand{\bkev}[2]{\Braket{#2|#1|#2}}



\DeclareMathOperator{\hV}{\hat{\vb{V}}}

\DeclareMathOperator{\hx}{\hat{\vb{x}}}
\DeclareMathOperator{\hy}{\hat{\vb{y}}}
\DeclareMathOperator{\hz}{\hat{\vb{z}}}

\DeclareMathOperator{\hA}{\hat{\vb{A}}}

\DeclareMathOperator{\hQ}{\hat{\vb{Q}}}
\DeclareMathOperator{\hI}{\hat{\vb{I}}}
\DeclareMathOperator{\psis}{\psi^\ast}
\DeclareMathOperator{\Psis}{\Psi^\ast}
\DeclareMathOperator{\hi}{\hat{\vb{i}}}
\DeclareMathOperator{\hj}{\hat{\vb{j}}}
\DeclareMathOperator{\hk}{\hat{\vb{k}}}
\DeclareMathOperator{\hr}{\hat{\vb{r}}}
\DeclareMathOperator{\hT}{\hat{\vb{T}}}
\DeclareMathOperator{\hH}{\hat{H}}
\DeclareMathOperator{\hh}{\hat{h}}               % helicity
\DeclareMathOperator{\hL}{\hat{\vb{L}}}
\DeclareMathOperator{\hp}{\hat{\vb{p}}}

\DeclareMathOperator{\ha}{\hat{\vb{a}}}
\DeclareMathOperator{\hS}{\hat{\vb{S}}}
\DeclareMathOperator{\hSigma}{\hat{\bm\Sigma}}
\DeclareMathOperator{\hJ}{\hat{\vb{J}}}
\DeclareMathOperator{\hP}{\hat{\vb{P}}}          % Parity
\DeclareMathOperator{\hC}{\hat{\vb{C}}} 
\DeclareMathOperator{\Tdv}{-\dfrac{\hbar^2}{2m}\dv[2]{x}}
\DeclareMathOperator{\Tna}{-\dfrac{\hbar^2}{2m}\nabla^2}
\DeclareMathOperator{\vna}{\vnabla}
\DeclareMathOperator{\nna}{\nabla^2}
\newcommand{\naCarExpd}[1]{\pdv[2]{#1}{x} + \pdv[2]{#1}{y} + \pdv[2]{#1}{z}}
\newcommand{\naCyl}{\qty[\dfrac{1}{\rho}\pdv{\rho}\qty(\rho\pdv{\rho}) + \dfrac{1}{\rho^2}\pdv[2]{\phi} + \pdv[2]{z}]}

%\DeclareMathOperator{\g#0}{\gamma^0}
%\DeclareMathOperator{\g1}{\gamma^1}
%\DeclareMathOperator{\g2}{\gamma^2}
%\DeclareMathOperator{\g3}{\gamma^3}
%\DeclareMathOperator{\g5}{\gamma^5}
\newcommand{\g}[1]{\gamma^{#1}}
\DeclareMathOperator{\gmuu}{\gamma^\mu}
\DeclareMathOperator{\gmud}{\gamma_\mu}
%\newcommand{\G}[2]{g^{#1#2}}

%% Code
\definecolor{codegray}{gray}{0.9}
\newcommand{\code}[1]{\colorbox{codegray}{\texttt{#1}}}

\newcommand{\subsbul}{\subsection*{$ \bullet $}}
\newcommand{\dis}{\displaystyle}
%\numberwithin{equation}{section}
%\allowdisplaybreaks[4]

\title{Daily Notes}
\author{王石嵘\\
%\vspace{5pt}\\
%161240065\\
wsr@smail.nju.edu.cn
}
\date{\today} % Activate to display a given date or no date (if empty),
         % otherwise the current date is printed 

\begin{document}
% \boldmath

\maketitle

\tableofcontents

\newpage

\section{06/23}
\subsection{PySCF dft routine}
in \code{dft.rks}
\begin{lstlisting}[language=python]
def get_veff(ks, mol=None, dm=None, dm_last=0, vhf_last=0, hermi=1):
    ...
    n, exc, vxc = ks._numint.nr_rks(mol, ks.grids, ks.xc, dm)
    # add vj,ecoul
    # add vk, HF-X if hyb
\end{lstlisting}
in \code{dft.numint}
\begin{lstlisting}[language=python]
def nr_rks(ni, mol, grids, xc_code, dms, relativity=0, hermi=0, max_memory=2000, verbose=None):
    ...
    xctype = ni._xc_type(xc_code)
    make_rho, nset, nao = ni._gen_rho_evaluator(mol, dms, hermi)
    
    shls_slice = (0, mol.nbas)
    ao_loc = mol.ao_loc_nr()
    
    nelec = numpy.zeros(nset)
    excsum = numpy.zeros(nset)
    vmat = numpy.zeros((nset,nao,nao))
    aow = None
    if xctype == 'LDA':
	    ao_deriv = 0
	    for ao, mask, weight, coords in ni.block_loop(mol, grids, nao, ao_deriv, max_memory):
	 	    aow = numpy.ndarray(ao.shape, order='F', buffer=aow)
	        for idm in range(nset):
		 	    rho = make_rho(idm, ao, mask, 'LDA')
		 	    exc, vxc = ni.eval_xc(xc_code, rho, 0, relativity, 1, verbose)[:2]
			    vrho = vxc[0]
			    den = rho * weight
			    nelec[idm] += den.sum()
			    excsum[idm] += (den * exc).sum()  # E_xc 
			    # *.5 because vmat + vmat.T
			    aow = numpy.einsum('pi,p->pi', ao, .5*weight*vrho, out=aow)
			    vmat[idm] += _dot_ao_ao(mol, ao, aow, mask, shls_slice, ao_loc)
			    rho = exc = vxc = vrho = None
\end{lstlisting}
thus
\begin{equation}\label{key}
E_{xc} = \sum_i \varepsilon_{xc}(\vb{r}_i)\rho(\vb{r}_i) w(\vb{r}_i)
\end{equation}

But how we allocate \code{exc, rho, weight} to each atom?\\
Let's try \code{dft.gen\_grid}
\begin{lstlisting}[language=python]
class Grids(lib.StreamObject):
    def build(self, mol=None, with_non0tab=False):
        atom_grids_tab = self.gen_atomic_grids(mol, self.atom_grid, self.radi_method, self.level, self.prune)
        self.coords, self.weights = self.gen_partition(mol, atom_grids_tab, self.radii_adjust, self.atomic_radii, self.becke_scheme)
        if with_non0tab:
            self.non0tab = self.make_mask(mol, self.coords)
        else:
            self.non0tab = None
        return self.coords, self.weights
\end{lstlisting}
\begin{lstlisting}[language=python]
def gen_partition(mol, atom_grids_tab, radii_adjust=None, atomic_radii=radi.BRAGG_RADII, 
                  becke_scheme=original_becke):
    atm_coords = numpy.asarray(mol.atom_coords() , order='C')
    atm_dist = radi._inter_distance(mol)
    #  if default settings
    def gen_grid_partition(coords):
        coords = numpy.asarray(coords, order='F')
        ngrids = coords.shape[0]
        pbecke = numpy.empty((mol.natm,ngrids))
        libdft.VXCgen_grid(pbecke.ctypes.data_as(ctypes.c_void_p),
                           coords.ctypes.data_as(ctypes.c_void_p),
                           atm_coords.ctypes.data_as(ctypes.c_void_p),
                           p_radii_table,
                           ctypes.c_int(mol.natm), ctypes.c_int(ngrids))
        return pbecke
    coords_all = []
    weights_all = []
    for ia in range(mol.natm):
        coords, vol = atom_grids_tab[mol.atom_symbol(ia)] # grid coords wrt atom coord
        coords = coords + atm_coords[ia] # get real coords
        pbecke = gen_grid_partition(coords) # do becke partition. 
        weights = vol * pbecke[ia] * (1./pbecke.sum(axis=0))
        coords_all.append(coords)
        weights_all.append(weights)
    return numpy.vstack(coords_all), numpy.hstack(weights_all)
\end{lstlisting}
Actually,
\begin{equation}\label{key}
E_{xc} = \sum_i \varepsilon_{xc}(\vb{r}_i)\rho(\vb{r}_i) V_i W_i
\end{equation}
\begin{equation}\label{key}
W_i = \dfrac{P_{i,A}}{\sum_A P_{i,A}}
\end{equation}
where $ W_i $ measures the extent to which a grid point belongs to some atom $ A $, in Becke partition. $ W_i = 1 $ when close to only one atom, and $ W_i = 0\sim 1 $ in other cases (i.e. shared). $ V_i $ is \code{vol} above.\\






\end{document}