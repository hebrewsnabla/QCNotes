\documentclass[UTF8]{ctexart} % use larger type; default would be 10pt

\usepackage[utf8]{inputenc} % set input encoding (not needed with XeLaTeX)

%%% Examples of Article customizations
% These packages are optional, depending whether you want the features they provide.
% See the LaTeX Companion or other references for full information.

%%% PAGE DIMENSIONS
\usepackage{geometry} % to change the page dimensions
\geometry{a4paper} % or letterpaper (US) or a5paper or....
% \geometry{margin=2in} % for example, change the margins to 2 inches all round
% \geometry{landscape} % set up the page for landscape
%   read geometry.pdf for detailed page layout information

\usepackage{graphicx} % support the \includegraphics command and options

% \usepackage[parfill]{parskip} % Activate to begin paragraphs with an empty line rather than an indent

%%% PACKAGES
\usepackage{booktabs} % for much better looking tables
\usepackage{array} % for better arrays (eg matrices) in maths
\usepackage{paralist} % very flexible & customisable lists (eg. enumerate/itemize, etc.)
\usepackage{verbatim} % adds environment for commenting out blocks of text & for better verbatim
\usepackage{subfig} % make it possible to include more than one captioned figure/table in a single float
% These packages are all incorporated in the memoir class to one degree or another...

%%% HEADERS & FOOTERS
\usepackage{fancyhdr} % This should be set AFTER setting up the page geometry
\pagestyle{fancy} % options: empty , plain , fancy
\renewcommand{\headrulewidth}{0pt} % customise the layout...
\lhead{}\chead{}\rhead{}
\lfoot{}\cfoot{\thepage}\rfoot{}

%%% SECTION TITLE APPEARANCE
\usepackage{sectsty}
\allsectionsfont{\sffamily\mdseries\upshape} % (See the fntguide.pdf for font help)
% (This matches ConTeXt defaults)

%%% ToC (table of contents) APPEARANCE
\usepackage[nottoc,notlof,notlot]{tocbibind} % Put the bibliography in the ToC
\usepackage[titles,subfigure]{tocloft} % Alter the style of the Table of Contents
\renewcommand{\cftsecfont}{\rmfamily\mdseries\upshape}
\renewcommand{\cftsecpagefont}{\rmfamily\mdseries\upshape} % No bold!

%%% END Article customizations

%%% The "real" document content comes below...

\setlength{\parindent}{0pt}
\usepackage{physics}
\usepackage{amsmath}
%\usepackage{symbols}
\usepackage{AMSFonts}
\usepackage{bm}
%\usepackage{eucal}
\usepackage{mathrsfs}
\usepackage{amssymb}


\DeclareMathOperator{\p}{\prime}
\DeclareMathOperator{\ti}{\times}
\DeclareMathOperator{\s}{\sum_{n=1}^{\infty}}
\newcommand{\dis}{\displaystyle}
\numberwithin{equation}{section}

\title{Notes of XU Guangxian QC}
\author{hebrewsnabla}
%\date{} % Activate to display a given date or no date (if empty),
         % otherwise the current date is printed 

\begin{document}
% \boldmath
\maketitle

\setcounter{section}{11}
\section{}
\subsection{}
\setcounter{subsubsection}{1}
\subsubsection{Solving Closed-shell HF Eq with Variational Method}
\begin{equation}\label{key}
\Psi=\dfrac{1}{\sqrt{N!}}
\mdet{\psi_1(1)\alpha(1) & \psi_1(2)\alpha(2) &\cdots &\psi_1(N)\alpha(N)\\
\psi_2(1)\alpha(1) & \psi_2(2)\alpha(2) &\cdots &\psi_2(N)\alpha(N)\\
\cdots&\cdots&\cdots&\cdots\\
\psi_p(1)\alpha(1) & \psi_p(2)\alpha(2) &\cdots &\psi_p(N)\alpha(N)\\
\psi_{p+1}(1)\beta(1) & \psi_{p+1}(2)\beta(2) &\cdots &\psi_{p+1}(N)\beta(N)\\
\cdots&\cdots&\cdots&\cdots\\
\psi_N(1)\beta(1) & \psi_N(2)\beta(2) &\cdots &\psi_N(N)\beta(N)}
\end{equation}
where $N=2p$.\\
\begin{equation}\label{key}
\hat{\vb{H}}=\sum_{i=1}^N \hat{\vb{h}}_i+\sum_{i<j}\hat{\vb{g}}_{ij}
\end{equation}
Thus
\begin{equation}\label{key}
E=\sum_i f_i + \sum_{i<j}(J_{ij}-K_{ij})
\end{equation}
where
\begin{equation}
f_i=\ev{\hat{\vb{h}}_i}{\phi_i}
\end{equation}
\begin{equation}\label{key}
J_{ij}=\ev{\hat{\vb{J}}_{j}}{\phi_i}=\bra{\phi_i \phi_j}\hat{\vb{g}}_{ij}\ket{\phi_i \phi_j}
\end{equation}
\begin{equation}\label{key}
K_{ij}=\delta(m_{s_i}m_{s_j})\ev{\hat{\vb{K}}_{j}}{\phi_i}=\bra{\phi_i \phi_j}\hat{\vb{g}}_{ij}\ket{\phi_j \phi_i}
\end{equation}
We need to minimize functional
\begin{equation}\label{key}
\begin{aligned}
W=E-\sum_{i<j}\delta(m_{s_i}m_{s_j})\varepsilon_{ij}\braket{\psi_i}{\psi_j}
\end{aligned}
\end{equation}
when $\psi_i\rightarrow\psi_i+\delta\psi_i$
\begin{equation}\label{key}
\begin{aligned}
\delta E &=\int\delta\psi_i^*(1)\hat{\vb{h}}(1)\psi_i(1)\dd \tau_1 + \int\psi_i^*(1)\hat{\vb{h}}(1)\delta\psi_i(1)\dd \tau_1\\
&+ \sum_j\big[\int(\delta\psi_i^*(1)\psi_j^*(2)\hat{\vb{g}}_{ij}\psi_i(1)\psi_j(2)+\psi_i^*(1)\psi_j^*(2)\hat{\vb{g}}_{ij}\delta\psi_i(1)\psi_j(2))\dd \tau_1\dd \tau_2\\
&-\delta(m_{s_i}m_{s_j})\int(\delta\psi_i^*(1)\psi_j^*(2)\hat{\vb{g}}_{ij}\psi_j(1)\psi_i(2)+\psi_i^*(1)\psi_j^*(2)\hat{\vb{g}}_{ij}\delta\psi_j(1)\psi_i(2))\dd \tau_1\dd \tau_2\big]\\
&=\int\delta\psi_i^*(1)
\Bigg\{\hat{\vb{h}}(1)\psi_i(1)\dd \tau_1
+\sum_j\Big[\int\psi_j^*(2)\hat{\vb{g}}_{ij}\psi_i(1)\psi_j(2)
-\delta(m_{s_i}m_{s_j})\int\psi_j^*(2)\hat{\vb{g}}_{ij}\psi_j(1)\psi_i(2)\Big]\dd \tau_1\dd \tau_2\Bigg\}\\
&+\int\delta\psi_i(1)
\Bigg\{\hat{\vb{h}}(1)\psi_i^*(1)\dd \tau_1
+\sum_j\Big[\int\psi_i^*(1)\psi_j^*(2)\hat{\vb{g}}_{ij}\psi_j(2)
-\delta(m_{s_i}m_{s_j})\int\psi_i^*(2)\hat{\vb{g}}_{ij}\psi_j^*(1)\psi_i(2)\Big]\dd \tau_1\dd \tau_2\Bigg\}\\
\end{aligned}
\end{equation}



\setcounter{subsection}{1}
\setcounter{subsubsection}{1}
\section{HF Roothaan Eq}
For N nuclei and n electrons, denote spatial orbitals as $\{\phi_i\}\qty(i=1,2,\cdots,\dfrac{n}{2})$, thus the wavefunction is
\begin{equation}\label{key}
\Psi_0 = \abs{\phi_1\alpha(1)\phi_1\beta(2)\cdots\phi_{\frac{1}{2}}\alpha(n-1)\phi_{\frac{1}{2}}\beta(n)}
\end{equation}
def:
\begin{equation}\label{key}
\mathbf{\hat{h}}_i = -\dfrac{1}{2}\nabla_i^2 - \sum_{s=1}^N \dfrac{Z_s}{r_{is}}
\end{equation}
\begin{equation}\label{key}
\mathbf{\hat{g}}=\dfrac{1}{r_{ij}}
\end{equation}
thus
\begin{equation}\label{key}
E=2\sum_i f_i + \sum_i^\frac{n}{2}\sum_j^\frac{n}{2}(2J_{ij}-K_{ij})
\end{equation}
where
\begin{equation}\label{key}
f_i=\ev{\hat{\vb{h}}_i}{\phi_i}
\end{equation}
\begin{equation}\label{key}
J_{ij}=\ev{\hat{\vb{J}}_{j}}{\phi_i}=\bra{\phi_i \phi_j}\hat{\vb{g}}_{ij}\ket{\phi_i \phi_j}
\end{equation}
\begin{equation}\label{key}
K_{ij}=\ev{\hat{\vb{K}}_{j}}{\phi_i}=\bra{\phi_i \phi_j}\hat{\vb{g}}_{ij}\ket{\phi_j \phi_i}
\end{equation}
Suppose
\begin{equation}\label{key}
\phi_i=\sum_\mu^m c_{\mu i}\chi_\mu
\end{equation}
thus
\begin{equation}\label{key}
f_i=\sum_\mu\sum_\nu c_{\mu i}^* c_{\nu i}h_{\mu\nu}
\end{equation}
\begin{equation}\label{key}
J_{ij}=\sum_\mu\sum_\lambda\sum_\nu\sum_\sigma c_{\mu i}^*c_{\lambda j} c_{\nu i}c_{\sigma j}(\mu\nu|\lambda\sigma)
\end{equation}
\begin{equation}\label{key}
K_{ij}=\sum_\mu\sum_\lambda\sum_\nu\sum_\sigma c_{\mu i}^*c_{\lambda j} c_{\nu i}c_{\sigma j}(\mu\sigma|\lambda\nu)
\end{equation}
\begin{equation}\label{key}
\braket{\phi_i}{\phi_j}=\sum_{\mu\nu}c_{\mu i}^* c_{\nu j}S_{\mu\nu}
\end{equation}
where
\begin{equation}\label{key}
h_{\mu\nu}=\int\chi_\mu(1)\hat{\vb{h}}(1)\chi_\nu(1)\dd \tau_1
\end{equation}
\begin{equation}\label{key}
(\mu\nu|\lambda\sigma)=\iint\chi_\mu(1)\chi_\nu(1)\hat{\vb{g}}_{12}\chi_\lambda(2)\chi_\sigma(2)\dd\tau_1\dd\tau_2
\end{equation}
\begin{equation}\label{key}
S_{\mu\nu}=\int\chi_\mu(1)\chi_\nu(1)\dd\tau_1
\end{equation}
\begin{equation}\label{key}
E=2\sum_\mu\sum_\nu\sum_i c_{\mu i}^* c_{\nu i}h_{\mu\nu}+\sum_\mu\sum_\lambda\sum_\nu\sum_\sigma\sum_{ij} c_{\mu i}^*c_{\lambda j} c_{\nu i}c_{\sigma j} [2(\mu\nu|\lambda\sigma)-(\mu\sigma|\lambda\nu)]
\end{equation}












\end{document}
