%\documentclass[UTF8]{ctexart} % use larger type; default would be 10pt
\documentclass[a4paper]{article}
\usepackage{xeCJK}
%\usepackage[utf8]{inputenc} % set input encoding (not needed with XeLaTeX)

%%% Examples of Article customizations
% These packages are optional, depending whether you want the features they provide.
% See the LaTeX Companion or other references for full information.

%%% PAGE DIMENSIONS
\usepackage{geometry} % to change the page dimensions
\geometry{a4paper} % or letterpaper (US) or a5paper or....
\geometry{margin=1in} % for example, change the margins to 2 inches all round
% \geometry{landscape} % set up the page for landscape
%   read geometry.pdf for detailed page layout information

\usepackage{graphicx} % support the \includegraphics command and options

% \usepackage[parfill]{parskip} % Activate to begin paragraphs with an empty line rather than an indent

%%% PACKAGES
\usepackage{booktabs} % for much better looking tables
\usepackage{array} % for better arrays (eg matrices) in maths
\usepackage{paralist} % very flexible & customisable lists (eg. enumerate/itemize, etc.)
\usepackage{verbatim} % adds environment for commenting out blocks of text & for better verbatim
\usepackage{subfig} % make it possible to include more than one captioned figure/table in a single float
% These packages are all incorporated in the memoir class to one degree or another...

%%% HEADERS & FOOTERS
\usepackage{fancyhdr} % This should be set AFTER setting up the page geometry
\pagestyle{fancy} % options: empty , plain , fancy
\renewcommand{\headrulewidth}{0pt} % customise the layout...
\lhead{}\chead{}\rhead{}
\lfoot{}\cfoot{\thepage}\rfoot{}

%%% SECTION TITLE APPEARANCE
\usepackage{sectsty}
\allsectionsfont{\sffamily\mdseries\upshape} % (See the fntguide.pdf for font help)
% (This matches ConTeXt defaults)

%%% ToC (table of contents) APPEARANCE
\usepackage[nottoc,notlof,notlot]{tocbibind} % Put the bibliography in the ToC
\usepackage[titles,subfigure]{tocloft} % Alter the style of the Table of Contents
\renewcommand{\cftsecfont}{\rmfamily\mdseries\upshape}
\renewcommand{\cftsecpagefont}{\rmfamily\mdseries\upshape} % No bold!

%%% END Article customizations

%%% The "real" document content comes below...

\setlength{\parindent}{0pt}
\usepackage{physics}
\usepackage{amsmath}
%\usepackage{symbols}
\usepackage{AMSFonts}
\usepackage{bm}
%\usepackage{eucal}
\usepackage{mathrsfs}
\usepackage{amssymb}
\usepackage{float}
\usepackage{multicol}
\usepackage{abstract}
\usepackage{empheq}
\usepackage{extarrows}
\usepackage{textcomp}
\usepackage{fontspec}
\usepackage{braket}
\usepackage{siunitx}
\usepackage{xcolor}
\usepackage{hyperref}
\usepackage{listings}
%\usepackage{tcolorbox}
\usepackage[T1]{fontenc}
\usepackage{beramono}
\usepackage{tikz}
\usepackage{mhchem}

\setmonofont[Mapping={}]{Consolas}	%英文引号之类的正常显示,相当于设置英文字体
%\setsansfont{Monaco} %设置英文字体 Monaco, Consolas,  Fantasque Sans Mono
%\setmainfont{Monaco} %设置英文字体
\definecolor{mygreen}{rgb}{0,0.6,0}
\definecolor{mygray}{rgb}{0.5,0.5,0.5}
\definecolor{mymauve}{rgb}{0.58,0,0.82}
\lstset{ 
	backgroundcolor=\color{white},   % choose the background color; you must add \usepackage{color} or \usepackage{xcolor}; should come as last argument
	basicstyle=\footnotesize\ttfamily,        % the size of the fonts that are used for the code
	breakatwhitespace=false,         % sets if automatic breaks should only happen at whitespace
	breaklines=true,                 % sets automatic line breaking
	captionpos=b,                    % sets the caption-position to bottom
	commentstyle=\color{mygreen},    % comment style
	deletekeywords={...},            % if you want to delete keywords from the given language
	escapeinside={\%*}{*)},          % if you want to add LaTeX within your code
	extendedchars=true,              % lets you use non-ASCII characters; for 8-bits encodings only, does not work with UTF-8
	firstnumber=1,                % start line enumeration with line 1000
	frame=single,	                   % adds a frame around the code
	keepspaces=true,                 % keeps spaces in text, useful for keeping indentation of code (possibly needs columns=flexible)
	keywordstyle=\color{blue},       % keyword style
	language=Octave,                 % the language of the code
	morekeywords={*,...},            % if you want to add more keywords to the set
	numbers=left,                    % where to put the line-numbers; possible values are (none, left, right)
	numbersep=5pt,                   % how far the line-numbers are from the code
	numberstyle=\tiny\color{mygray}, % the style that is used for the line-numbers
	rulecolor=\color{black},         % if not set, the frame-color may be changed on line-breaks within not-black text (e.g. comments (green here))
	showspaces=false,                % show spaces everywhere adding particular underscores; it overrides 'showstringspaces'
	showstringspaces=false,          % underline spaces within strings only
	showtabs=false,                  % show tabs within strings adding particular underscores
	stepnumber=2,                    % the step between two line-numbers. If it's 1, each line will be numbered
	stringstyle=\color{mymauve},     % string literal style
	tabsize=2,	                   % sets default tabsize to 2 spaces
	title=\lstname                   % show the filename of files included with \lstinputlisting; also try caption instead of title
}

\sisetup{
	separate-uncertainty = true,
	inter-unit-product = \ensuremath{{}\cdot{}}
}

\DeclareMathOperator{\p}{\prime}
\DeclareMathOperator{\ti}{\times}
\DeclareMathOperator{\intinf}{\int_0^\infty}
\DeclareMathOperator{\intdinf}{\int_{-\infty}^\infty}
\DeclareMathOperator{\intzpi}{\int_0^\pi}
\DeclareMathOperator{\intztpi}{\int_0^{2\pi}}
\DeclareMathOperator{\suminf}{\sum_{n=1}^{\infty}}
\DeclareMathOperator{\suminfz}{\sum_{n=0}^\infty}
\DeclareMathOperator{\sumkinf}{\sum_{k=1}^{\infty}}
\DeclareMathOperator{\sumkinfz}{\sum_{k=0}^\infty}
\DeclareMathOperator{\e}{\mathrm{e}}
\DeclareMathOperator{\I}{\mathrm{i}}
\DeclareMathOperator{\Arg}{\mathrm{Arg}}
\DeclareMathOperator{\ra}{\rightarrow}
\DeclareMathOperator{\llra}{\longleftrightarrow}
\DeclareMathOperator{\lra}{\longrightarrow}
\DeclareMathOperator{\dlra}{\Leftrightarrow}
\DeclareMathOperator{\dra}{\Rightarrow}
\newcommand{\bkk}[1]{\Braket{#1|#1}}
\newcommand{\bk}[2]{\Braket{#1|#2}}
\newcommand{\bkev}[2]{\Braket{#2|#1|#2}}



\DeclareMathOperator{\hV}{\hat{\vb{V}}}

\DeclareMathOperator{\hx}{\hat{\vb{x}}}
\DeclareMathOperator{\hy}{\hat{\vb{y}}}
\DeclareMathOperator{\hz}{\hat{\vb{z}}}

\DeclareMathOperator{\hA}{\hat{\vb{A}}}

\DeclareMathOperator{\hQ}{\hat{\vb{Q}}}
\DeclareMathOperator{\hI}{\hat{\vb{I}}}
\DeclareMathOperator{\psis}{\psi^\ast}
\DeclareMathOperator{\Psis}{\Psi^\ast}
\DeclareMathOperator{\hi}{\hat{\vb{i}}}
\DeclareMathOperator{\hj}{\hat{\vb{j}}}
\DeclareMathOperator{\hk}{\hat{\vb{k}}}
\DeclareMathOperator{\hr}{\hat{\vb{r}}}
\DeclareMathOperator{\hT}{\hat{\vb{T}}}
\DeclareMathOperator{\hH}{\hat{H}}
\DeclareMathOperator{\hh}{\hat{h}}               % helicity
\DeclareMathOperator{\hL}{\hat{\vb{L}}}
\DeclareMathOperator{\hp}{\hat{\vb{p}}}

\DeclareMathOperator{\ha}{\hat{\vb{a}}}
\DeclareMathOperator{\hS}{\hat{\vb{S}}}
\DeclareMathOperator{\hSigma}{\hat{\bm\Sigma}}
\DeclareMathOperator{\hJ}{\hat{\vb{J}}}
\DeclareMathOperator{\hP}{\hat{\vb{P}}}          % Parity
\DeclareMathOperator{\hC}{\hat{\vb{C}}} 
\DeclareMathOperator{\Tdv}{-\dfrac{\hbar^2}{2m}\dv[2]{x}}
\DeclareMathOperator{\Tna}{-\dfrac{\hbar^2}{2m}\nabla^2}
\DeclareMathOperator{\vna}{\vnabla}
\DeclareMathOperator{\nna}{\nabla^2}
\newcommand{\naCarExpd}[1]{\pdv[2]{#1}{x} + \pdv[2]{#1}{y} + \pdv[2]{#1}{z}}
\newcommand{\naCyl}{\qty[\dfrac{1}{\rho}\pdv{\rho}\qty(\rho\pdv{\rho}) + \dfrac{1}{\rho^2}\pdv[2]{\phi} + \pdv[2]{z}]}

%\DeclareMathOperator{\g#0}{\gamma^0}
%\DeclareMathOperator{\g1}{\gamma^1}
%\DeclareMathOperator{\g2}{\gamma^2}
%\DeclareMathOperator{\g3}{\gamma^3}
%\DeclareMathOperator{\g5}{\gamma^5}
\newcommand{\g}[1]{\gamma^{#1}}
\DeclareMathOperator{\gmuu}{\gamma^\mu}
\DeclareMathOperator{\gmud}{\gamma_\mu}
%\newcommand{\G}[2]{g^{#1#2}}

%% Code
\definecolor{codegray}{gray}{0.9}
\newfontfamily\Consolas{Consolas}
\newcommand{\code}[1]{\colorbox{codegray}{{\Consolas#1}}}
\lstnewenvironment{mcode}
{\lstset{backgroundcolor=\color{lightgray},
		 xleftmargin=0.5cm,
		 frame=tlbr,framesep=4pt,framerule=0pt,
		 language=python,
		 keepspaces=false,
		 numbers=none
}}%
{}
\newcommand{\subsbul}{\subsection*{$ \bullet $}}
\newcommand{\dis}{\displaystyle}
%\numberwithin{equation}{section}
%\allowdisplaybreaks[4]

\title{Intro to Polymer Science}
\author{王石嵘\\
%\vspace{5pt}\\
%161240065\\
wsr@smail.nju.edu.cn
}
\date{\today} % Activate to display a given date or no date (if empty),
         % otherwise the current date is printed 

\begin{document}
% \boldmath

\maketitle

\tableofcontents

\newpage

Prof. Ed Palermo, Rensselaer Polytechnic Institute\\
\section{Introduction}
\subsection{Brief History}
"gun cotton", using cellulose, 1843\\
Vulcanized rubber, 1844\\
First fully-synthetic: Bakelite, 1907\\
First linear fully-synthetic: Nylon, 1937, W. Carothers\\

H. Wieland -- large mol is junk\\
H. Staudinger -- No! (article: \"Uber Polymerisation)\\

polystyrene -- most difficult to recycle, (all landfill in US)\\
substitute: polylactic acid (decompose in land)

How large is large? What's the cutoff?

\subsection{Classes}
\begin{enumerate}
	\item Thermosets (much cross-linked, stable under heat)
	\item Thermoplastics (no cross-linked, melt easily)
	\begin{enumerate}
		\item Semi-crystalline (Note: No fully crystallized polymer)
		\item Amorphous
	\end{enumerate}
	\item Elastomers (little cross-linked)
\end{enumerate}

\subsection{Molecular Weight Distribution}



\section{Polymer Chain Conformations}
\subsection{Freely Joint Chain}
1D free walk of R. Feynman\\
\begin{equation}\label{key}
R_1^2 = \ell^2
\end{equation}
\begin{equation}\label{key}
R_N = R_{N-1}+\ell, R_{N-1}-\ell
\end{equation}
\begin{equation}\label{key}
\ev{R_N^2} = R_{N-1}^2 + \ell^2 = ... = N\ell^2 
\end{equation}
\begin{equation}\label{key}
\sqrt{\ev{R_N^2}} = \ell\sqrt{N}
\end{equation}
other ways to derive that:
\begin{enumerate}
	\item radial Gaussian distribution function
\end{enumerate}
\subsection{Real Chains}
def: characteristic ratio $ C_\infty $
\begin{equation}\label{key}
\sqrt{R_N^2} = \ell\sqrt{C_\infty N}
\end{equation}
\subsubsection{Free Rotation}
\begin{equation}\label{key}
\sqrt{R_N^2} = \ell\sqrt{N} \sqrt{\dfrac{1+\cos\theta}{1-\cos\theta}}
\end{equation}
where $ \theta = 180\textdegree - 109.47\textdegree $, thus
\begin{equation}\label{key}
\sqrt{R_N^2} = \ell\sqrt{2 N}
\end{equation}
\subsubsection{Hindered Rotation}
\begin{equation}\label{key}
\sqrt{R_N^2} = \ell\sqrt{2 N}\sqrt{\dfrac{1+\ev{\cos\phi}}{1-\ev{\cos\phi}}}
\end{equation}
\begin{table}[H]
	\centering
	\begin{tabular}{cccc}
		\hline
		$ \phi $ & $ \cos\phi $ & $ E_i $ & $ p(E_i) $\\ \hline
		-120 &&&\\
		0&&&\\
		120&&&\\ \hline
	\end{tabular}
\end{table}
\begin{equation}\label{key}
\sqrt{\ev{R_N^2}} = \ell\sqrt{6.9 N}
\end{equation}
%\subsubsection{Independent Rotation}
\subsubsection{The Kuhn Length}
def: Kuhn segment length $ \ell_k $
\begin{equation}\label{key}
\sqrt{R^2} = \ell_k\sqrt{N_k}
\end{equation}

\subsection{Scaling Laws for the SAW}
\begin{equation}\label{key}
\sqrt{\ev{R^2}} = \ell(C\cdot N)^\nu
\end{equation}

\section{Crystallization, Melting, and Glass Transition}

\subsection{Nucleation}

\subsection{Thermodynamics}
critical lamellar thickness
\begin{equation}\label{key}
l^* = \dfrac{4\sigma_{face}}{\Delta g_{vol}} \propto \dfrac{1}{\Delta T}
\end{equation}
where $ \Delta T = T_m - T $\\
crystallization requires $ l > l^* $.\\

Why polymer crystallize in a chain folding manner, rather than an extended chain? -- kinetics.


\subsection{Melting Temperature}
Why polymers melt over a range of temerature?\\
factors that affect $ T_m $
\begin{enumerate}
	\item backbone stiffness
	\item non-covalent interaction\\
	VdW, dipole-dipole (PVC), H-bonding (Nylon 6)
	
\end{enumerate}

\subsection{Glass Transition Temp.}
factors that affect $ T_g $
\begin{enumerate}
	\item molecular weight
	\begin{equation}\label{key}
	T_g = T_{g\infty} - \dfrac{K}{M_n}
	\end{equation}
	\item backbone stiffness
	\item Side Chains
	\item Non-covalent interactions
	\item corss linking
	\item Dilutents/Plasticizer
	\item copolymerization and blending\\
	blend miscibility
\end{enumerate}

\section{Thermodynamics of Polymer Phase Separation}
\subsection{Entropy of Mixing}

\subsection{Enthalpy of Mixing}
\begin{equation}\label{key}
\Delta H_m = n_A f_B z \Delta\varepsilon_{AB}
\end{equation}
def: interaction parameter
\begin{equation}\label{key}
\chi = \dfrac{z\Delta\varepsilon_{AB}}{kT}
\end{equation}
Flory-Huggins Eq.
\begin{equation}\label{key}
\Delta G_m = \Delta H_m - T\Delta S_m
\end{equation}
\begin{equation}\label{key}
\dfrac{\Delta G}{kT} = n_A\Phi_B\chi + n_A\ln\Phi_A + n_B\ln\Phi_B
\end{equation}
$ m=DP $ for an approximation.

UCST: Upper critical solution temperature\\
Since $ \chi \sim 1/T $, we can calculate $ \chi_c $ at $ T_c $
\begin{equation}\label{key}
\chi_c = \dfrac{1}{2}\qty(\dfrac{1}{\sqrt{m_A}} + \dfrac{1}{\sqrt{m_B}})^2
\end{equation}
for polymer solution
\begin{equation}\label{key}
\chi_c = \dfrac{1}{2}\qty(1 + \dfrac{1}{\sqrt{m_p}})^2 \ra \dfrac{1}{2}
\end{equation}
$ \chi > \chi_c \;\dra$ can not mix at all\\ 


\end{document}