\documentclass[UTF8]{ctexart} % use larger type; default would be 10pt

\usepackage[utf8]{inputenc} % set input encoding (not needed with XeLaTeX)

%%% Examples of Article customizations
% These packages are optional, depending whether you want the features they provide.
% See the LaTeX Companion or other references for full information.

%%% PAGE DIMENSIONS
\usepackage{geometry} % to change the page dimensions
\geometry{a4paper} % or letterpaper (US) or a5paper or....
% \geometry{margin=2in} % for example, change the margins to 2 inches all round
% \geometry{landscape} % set up the page for landscape
%   read geometry.pdf for detailed page layout information

\usepackage{graphicx} % support the \includegraphics command and options

% \usepackage[parfill]{parskip} % Activate to begin paragraphs with an empty line rather than an indent

%%% PACKAGES
\usepackage{booktabs} % for much better looking tables
\usepackage{array} % for better arrays (eg matrices) in maths
\usepackage{paralist} % very flexible & customisable lists (eg. enumerate/itemize, etc.)
\usepackage{verbatim} % adds environment for commenting out blocks of text & for better verbatim
\usepackage{subfig} % make it possible to include more than one captioned figure/table in a single float
% These packages are all incorporated in the memoir class to one degree or another...

%%% HEADERS & FOOTERS
\usepackage{fancyhdr} % This should be set AFTER setting up the page geometry
\pagestyle{fancy} % options: empty , plain , fancy
\renewcommand{\headrulewidth}{0pt} % customise the layout...
\lhead{}\chead{}\rhead{}
\lfoot{}\cfoot{\thepage}\rfoot{}

%%% SECTION TITLE APPEARANCE
\usepackage{sectsty}
\allsectionsfont{\sffamily\mdseries\upshape} % (See the fntguide.pdf for font help)
% (This matches ConTeXt defaults)

%%% ToC (table of contents) APPEARANCE
\usepackage[nottoc,notlof,notlot]{tocbibind} % Put the bibliography in the ToC
\usepackage[titles,subfigure]{tocloft} % Alter the style of the Table of Contents
\renewcommand{\cftsecfont}{\rmfamily\mdseries\upshape}
\renewcommand{\cftsecpagefont}{\rmfamily\mdseries\upshape} % No bold!

%%% END Article customizations

%%% The "real" document content comes below...

\setlength{\parindent}{0pt}
\usepackage{physics}
\usepackage{amsmath}
%\usepackage{symbols}
\usepackage{AMSFonts}
\usepackage{bm}
%\usepackage{eucal}
\usepackage{mathrsfs}
\usepackage{amssymb}
\usepackage{float}
\usepackage{multicol}
\usepackage{abstract}
\usepackage{empheq}
\usepackage{extarrows}
\usepackage{textcomp}


\DeclareMathOperator{\p}{\prime}
\DeclareMathOperator{\ti}{\times}
\DeclareMathOperator{\intinf}{\int_0^\infty}
\DeclareMathOperator{\intdinf}{\int_{-\infty}^\infty}
\DeclareMathOperator{\suminf}{\sum_{n=1}^{\infty}}
\DeclareMathOperator{\suminfz}{\sum_{n=0}^\infty}
\DeclareMathOperator{\e}{\mathrm{e}}
\renewcommand{\I}{\mathrm{i}}
\DeclareMathOperator{\Arg}{\mathrm{Arg}}
\DeclareMathOperator{\ra}{\rightarrow}
\DeclareMathOperator{\llra}{\longleftrightarrow}
\DeclareMathOperator{\lra}{\longrightarrow}
\DeclareMathOperator{\dlra}{\Leftrightarrow}
\DeclareMathOperator{\dra}{\Rightarrow}

\newcommand{\dis}{\displaystyle}
\newcommand{\td}{\textdegree}
\numberwithin{equation}{section}
%\allowdisplaybreaks[4]

\title{CHEN Minbo, Molecular Dynamics}
\author{匡院 王石嵘\\
161240065}
\date{\today} % Activate to display a given date or no date (if empty),
         % otherwise the current date is printed 

\begin{document}
% \boldmath
\maketitle

\tableofcontents

\newpage

\section{}

\section{PES}

\section{MD}

Steps of MD simulation:
\begin{enumerate}
	\item Calculate potential energy $ U(\vb{q}) $
	\item Calculate forces $ \{\vb{F}_i\} $ and time evolution of nuclear position $ \{\vb{q}_i\} $
	\item Find other mechanic quantities
\end{enumerate}

\subsection{Integration Methods}
Gibson, Euler -- not accurate enough\\
Runge-Kutta -- too costly\\

\subsubsection{Verlet Method}
\begin{equation}\label{key}
\vb{F}_i(t) = -\nabla_{\vb{r}_i} U(\{\vb{r}_i(t)\})
\end{equation}
\begin{equation}\label{key}
\vb{r}_i(t+\Delta t) = \vb{r}_i(t) + \vb{v}_i(t)\tau + \dfrac{1}{2}\dfrac{\vb{F}_i(t)}{m_i}\tau^2 + \dfrac{1}{6}\vb{b}_i(t)\tau^3 + \mathcal{O}(\tau^4)
\end{equation}
\begin{equation}\label{key}
\vb{r}_i(t-\Delta t) = \vb{r}_i(t) - \vb{v}_i(t)\tau + \dfrac{1}{2}\dfrac{\vb{F}_i(t)}{m_i}\tau^2 - \dfrac{1}{6}\vb{b}_i(t)\tau^3 + \mathcal{O}(\tau^4)
\end{equation}
thus
\begin{equation}\label{key}
\vb{r}_i(t+\Delta t) = 2\vb{r}_i(t) - \vb{r}_i(t-\Delta t) + \dfrac{\vb{F}_i(t)}{m_i}\tau^2 + \mathcal{O}(\tau^4)
\end{equation}
i.e.
\begin{equation}\label{key}
r_{n+1} = 2 r_n - r_{n-1} + \dfrac{F_n}{m}\tau^2 + \mathcal{O}(\tau^4)
\end{equation}
\begin{equation}\label{key}
\begin{aligned}
v_n &= \dfrac{r_{n+1} - r_{n-1}}{2\tau} - \dfrac{1}{6}\vb{b}_n\tau^2\\
&= \dfrac{r_{n+1} - r_{n-1}}{2\tau} + \mathcal{O}(\tau^2)
\end{aligned}
\end{equation}
Verlet method don't need $ v $, but how to get $ r_{1} $?\\
by estimation
\begin{equation}\label{key}
r_1 = r_0 + v_0\tau + \dfrac{1}{2}\dfrac{F_0}{m}\tau^2 +\mathcal{O}\tau^3
\end{equation}

How to set $ \tau $?\\
$ \tau \; <$ characteristic time $ \ti\dfrac{1}{10} $\\
\begin{tabular}{|l|l|}
	\hline
	motions & characteristic time\\
	\hline
	local motions
\end{tabular}





\end{document}